\section{HTML}
\label{html}

\subsection{Was ist HTML?}
HTML steht für \textit{Hypertext Markup Language} und ist der grundlegende Baustein um eine Webanwendung zu erstellen.
HTML ist eine Beschreibungssprache, mit der sich die logischen Strukturen eines Dokuments beschreiben lassen.
Solche wären zum Beispiel: Kapitel, Unterkapitel, Absätze und eingebundene Bilder. Dazu verwendet man verschieden von HTML zur Verfügung gestellte
Befehle. Der Browser verarbeitet dann den geschriebenen Text sowie die Befehle und wandelt diese in eine grafische Oberfläche um.
Die Hypertext Markup Language wird ständig weiterentwickelt und befindet sich im momentan in der Version 5.2.

\subsection{Die Geschichte von HTML}
Da das Internet immer populärer wurde und somit immer mehr Daten veröffentlicht wurden, brauchte man allgemeine Regeln,
um ein Chaos zu vermeiden. Das Ziel von HTML war es, Dokumente und Daten unabhängig von der eingesetzten Hard- und Software anzeigen zu lassen.
Die erste HTML-Version war für die Strukturierung und Darstellung von wissenschaftlichen Informationen gedacht und wurde 1989 von \textit{Tim Berners-Lee}
entwickelt.

\cite{geschichteHTML}
