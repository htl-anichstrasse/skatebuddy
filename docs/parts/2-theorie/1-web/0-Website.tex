\section{Website}
\label{websitetheorie}

Eine Website ist ein Dokument im HTML-Format, welches sich durch einen Browser anzeigen lässt.
Jede Website besitzt eine URL (Uniform Resource Locator) mit welcher die Website erreichbar ist.
Die Start- bzw. Hauptseite besitzt hier meist die URL \textit{www.website.com}. Seitenabschnitte 
welche auf ein bestimmtes Verzeichnis hinweisen, haben am Ende der URL meist den Namen des 
Verzeichnisses stehen. Diese nennt man Landingpages Die würde dann wie folgt aussehen $\rightarrow$ 
\textit{www.website.com/Verzeichnis}. Das Gesamte einer Website, also mit allen Landingpages 
gemeinsame wird als \textbf{Website} bezeichnet. Auf Einer Website befindet sich nicht nur HTML, 
sondern auch die Gestaltungssprache CSS. Diese ist dafür zuständig dem HTML-Dokument ein 
anschauliches Aussehen zu geben.~\cite{Website}