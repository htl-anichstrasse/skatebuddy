\subsection{React}
\label{reactjs}
React ist eine der beliebtesten JavaScript Frameworks zum Erstellen von sogenannten Single-Page-Apps. React beschreibt sich selbst als 
JavaScript Bibliothek und nicht wie andere als ein Framework. React ist viel mehr eine Anleitung, wie man verschieden Dinge machen könnte.
React ist also kein All-in-One Paket.


Reacts Entwicklungsgrund war es eine möglichst performante Oberfläche zu gestalten. Dabei werden einfache Views erstellt und React
übernimmt die Änderung der Daten innerhalb der View. React erstellt einen sogenannten virtuellen DOM bei der Datenanzeige.


\begin{figure}[H]
  \begin{center}
    \includegraphics[width=0.1\textwidth]{Theorie/React/React-icon.svg.png}
    \caption{ReactJS Logo~\cite{reactjs}}
  \end{center}
\end{figure}

\subsubsection{Virtuelles DOM}
VDOM ist ein Programmierkonzept, bei dem durch den Prozess \textit{Reconciliation} eine ideale, virtuelle Darstellung der Benutzeroberfläche
im Speicher gehalten wird und mit dem realen DOM durch eine Bibliothek wie ReactDOM synchronisiert wird. Dies ermöglicht eine deklarative
API von React, bei der man den Zustand der Benutzeroberfläche angibt und React stellt sicher, dass das DOM diesem Zustand entspricht. Dies 
entfernt verschiedene Arbeitsschritte, welche sie sonst durchführen müssten.
VDOM ist viel mehr ein Muster als eine spezifische Technologie. Der Begriff virtuelles DOM wird in React oft mit Objekten in Verbindung gebracht,
da diese die Benutzeroberfläche darstellen. Um jedoch zusätzliche Information über den Komponentenbaum zu speichern benutzt React interne 
Objekte nämlich sogenannte \textit{Fiber}. 
\cite{DOM}

\subsubsection{Single-Page-Apps}
Wenn man bei einer Webanwendung sich die View ändert, wird dies oft mit eine Seitenwechsel assoziiert. Dies ist zwar nicht falsch, da 
frühere Webanwendung in der tat solch einen Seitenwechsel durchführten. Dies war jedoch sehr ineffizient, da bei einem Wechsel der View 
immer eine neue Seite geladen werden musste. Eine modernere Herangehensweise an diese Sache sind \textit{Single-Page-Apps (SPA)}.
Sie bestehen aus nur einer Seite mit mehreren Views. Anstatt dass bei einem View Wechsel eine neue Seite geladen wird, änderst sich je nach
State die aktive Seite. \cite{SPA}


\subsubsection{States}
Die Stärke in React liegt in den States. Ein State ist der Istzustand der Anwendung. Ändert sich die View der Seite, ändert sich auch der
State. Sobald sich irgendetwas auf der Seite ändert, ändert sich der State der Seite.


\subsubsection{Komponenten}
In React bestehen die Views aus sogenannten Komponenten. Mit React kann man vordefinierte CodeBits programmieren, welche anschließend zu einer
View zusammengesetzt werden. Eine React Anwendung besteht also aus logisch getrennten Einheiten, die unabhängig voneinander behandelt werden-
Bei React einzigartig ist, dass kein Pattern mit MVC oder MVVM verwendet wird. In React Logik als sowohl auch Template in einer JS Datei
verwaltet.
HTML-Code wird in React in einer JavaScript Datei geschrieben im sogenannten \textit{JSX-Format}. JSX ist HTML jedoch sehr ähnlich.


In React gibt es zwei verschiedene Arten eine Komponente zu erstellen.

TODO:

\cite{Komp}
