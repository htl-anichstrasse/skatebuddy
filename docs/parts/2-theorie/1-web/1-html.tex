\section{HTML}
\label{html}

\subsection{Was ist HTML?}
HTML steht für \textit{Hypertext Markup Language} und ist der grundlegende Baustein um eine
Internetanwendung zu erstellen. HTML ist eine Beschreibungssprache, mit der sich die logischen
Strukturen eines Dokuments beschreiben lassen. Solche wären zum Beispiel: Kapitel, Unterkapitel,
Absätze und eingebundene Bilder. Dafür verwendet man verschiedene von HTML zur Verfügung gestellte
Befehle. Der Browser verarbeitet den geschriebenen Text sowie die Befehle und wandelt diese in
eine grafische Oberfläche um.

\subsection{Die Geschichte von HTML}
Da das Internet immer populärer wurde und somit immer mehr Daten im Internet veröffentlicht wurden, 
brauchte man allgemeine Regeln um ein Chaos zu vermeiden. Das Ziel von HTML war es, Dokumente und 
Daten unabhängig von der eingesetzten Hard- und Software anzeigen zu lassen. 
Die erste HTML-Version war für die Strukturierung und Darstellung von wissenschaftlichen 
Informationen gedacht und wurde 1989 von \textit{Tim Berners-Lee} entwickelt. Die Hypertext Markup 
Language wird ständig weiterentwickelt und befindet sich im Moment in der Version 5.2. 
~\cite{geschichteHTML}
\\
\\
\\
HTML hat sich in den letzten Jahren wie folgt entwickelt:
\begin{center}
    \begin{table}[]
        \centering
        \begin{tabular}{|l|l|l|} \hline
            {\textbf{Jahr}} & {\textbf{Version}} & {\textbf{Neues Feature}} \\ \hline
            1993                & HTML                   & Text und Bildgeneration      \\ \hline
            1995                & HTML 2.0               & Formulare                    \\ \hline
            1997                & HTML 3.2               & Tabellen, Applets            \\ \hline
            1997                & HTML 4.0               & CSS                          \\ \hline
            2014                & HTML 5                 & Neues Vokabular              \\ \hline
            2017                & HTML 5.2               & Neue Features               \\ \hline
        \end{tabular}
    \end{table}
\end{center}