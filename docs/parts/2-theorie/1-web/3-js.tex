\section{JavaScript}
\label{js}

JavaScript ist eine Programmiersprache/Skriptsprache, die üblicherweise in Webseiten Verwendung 
findet. Sie wird jedoch auch noch in vielen anderen Umgebungen außerhalb der Website Entwicklung
benutzt, wie zum Beispiel \nameref{nodejs}. 
JavaScript besitzt First-Class-Functions (Funktionen erster Klasse). 
Außerdem is JS eine prototypbasierte Sprache, welche mehreren Paradigmen folgt und dynamisch als sowohl auch 
objektorientiert ist. JavaScript ist plattformunabhängig. Zusätzlich ist es sehr kompakt und 
ressourcenschonend. JavaScript sollte man nicht mit Java verwechseln. 
Sie besitzen beide eine unterschiedliche Syntax, Semantik und Verwendung.~\cite{JS}

\subsection{First-Class-Functions}
Funktionen erster Klasse werden wie Variablen behandelt. Bei einer Programmiersprache welche 
First-Class-Funktionen besitzt, kann man Funktionen als Parameter übergeben, einer Variable 
zuweisen oder von einer anderen Funktion zurückgegeben werden.~\cite{First-Class-Funktion}

\subsubsection{Beispiel für die Zuweisung einer Funktion an eine Variable}
\begin{lstlisting}
    const v = function(){
        Console.log("Ausgaben in der Konsole")
    }

    v();
\end{lstlisting}
Hier wird der Variable \textit{v} eine anonyme Funktion zugewiesen, welche in der Konsole 
eine Ausgabe liefert. Diese Funktion wird dann ganz einfach über die Variable mittels den zwei 
Klammern aufgerufen.


\subsubsection{Beispiel für das Übergeben einer Funktion als Argument}
\begin{lstlisting}
    function sagHallo() {
        return "Hallo, ";
    }
    function gruessen(gruss, name){
        Console.log(gruss(), name);
    }

    gruessen(sagHallo(), "JavaScript!");
\end{lstlisting}
Hier übergeben wir der Funktion \textit{gruessen} 2 Parameter. Einer dieser beiden Parameter ist eine
Funktion namens \textit{sagHallo}, welche ``Hallo, `` als return liefert. Führen wir nun 
die Methode gruessen mit diesen Parametern aus, wird in die Console der Satz 
\textit{``Hallo, Javascript``} geschrieben.

\subsubsection{Eine Funktion als Return}
\begin{lstlisting}
    function sagHallo(){
        return function(){
            Console.log("Hallo")
        }
    }
\end{lstlisting}
Hier wird ganz einfach als return-Wert eine Funktion übergeben.

\subsection{Objektorientierte Programmierung}
Objektorientierte Programmierung (OOP) eignet sich gut für große und komplexe Software, welche aktiv 
aktualisiert oder gewartet werden muss. OOP konzentriert sich nicht auf die Logik, sondern auf die 
Objekte, mit denen das Programm interagieren soll. Der Aufbau für das Programm bei der 
Objektorientierten Entwicklung ist auch für das Entwickeln von Vorteil. Andere Vorteile von OOP 
sind die Wiederverwendbarkeit von Code, Skalierbarkeit und die Effizienz.~\cite{OOP}

\subsubsection{Prinzipien von OOP}
\begin{description}
\item[Verkapselung]\hfill \\
 Objekte werden privat innerhalb einer definierten Grenze oder Klasse gehalten und implementiert. 
 Andere Objekte haben keinen Zugriff auf diese Klasse und keine Berechtigung
 Änderungen vorzunehmen. Dies versichert eine größere Programmiersicherheit.
\item[Abstraktion]\hfill \\
Objekte verbergen den unnötigen Implementierungscode und offenbaren nur interne Mechanismen, welche 
für andere Objekte relevant sind. Hilfreich für Entwickler um Änderungen und Ergänzungen vorzunehmen.
\item[Inheritance]\hfill \\
Man kann Beziehungen und Unterklassen zwischen Objekten zuweisen, um gemeinsame Logik 
wiederverwenden zu können. Vorteile hiervon sind eine verkürzte Entwicklungszeit und 
es sorgt für eine höhere Genauigkeit.
\end{description}

\subsubsection{Prototypbasierte Programmierung}
Ist eine andere Art von objektorientierter Programmierung, bei der keine Klassen verwendet werden. 
Stattdessen erzeugt man Objekte durch das Kopieren von bereits existierenden Objekten (Prototypen). 
Beim Kopieren werden alle Eigenschaften des kopierten Objekts übernommen. 
Man kann diese auch verändern und/oder ergänzen.