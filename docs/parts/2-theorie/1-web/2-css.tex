\section{CSS}
\label{css}

\subsection{Was ist CSS?}
Mit CSS \textit{(Cascading Style Sheets)} kann das Aussehen von HTML-Dokumenten bestimmt werden. 
CSS beeinflusst also um keiner Weiße den Inhalt der Seite, sondern nur das Design. Der Anwender 
kann mit CSS also die Schriftarten, Farbe, Höhen und Breiten einer Website definieren.

\subsection{Die Geschichte von CSS}
1994 wurde CSS erstmals von \textit{Håkon Wium Lie} vorgeschlagen. Zu dieser Zeit arbeitete
\textit{Bert Bos} an ein Darstellungsprogramm names \textit{Argo}, welcher dabei seine eigene 
Stilvorlagensprache benutzte. Die beiden taten sich letztendlich zusammen und entwickelten CSS. 
CSS war jedoch nicht die einzige Sprache, welche das Ziel hatte die Darstellung von HTML Dokumenten zu 
verändern. Sie war jedoch die erste Sprache, welche Regeln definierte, die über mehrere Stilvorlagen 
hinweg vererbt werden konnten. Das W3C (W3C ist das Gremium zur Standardisierung der Techniken im 
World Wide Web) wurde erstmals auf CSS aufmerksam bei einer Präsentation von Håkon und Bos an der 
\glqq Mosaic and the Web\grqq{} Konferenz in Chicago 1995. Sie arbeiteten zusammen mit anderen 
Mitgliedern weiter an CSS und am Dezember 1996 wurde die \textit{CSS Level 1 Recommendation} 
publiziert. Viele aktuell gängige Darstellungsprogramme benutzen zurzeit CSS 2.1. Diese Version wurde am 
7. Juni endgültig vom W3C empfohlen. Voraussetzung für diese Empfehlung war, dass es 
pro Merkmal mindestens zwei Programme gibt, welche es korrekt interpretieren können.


CSS Level 3 ist derzeit in Entwicklung und ist im Gegensatz zu den Vorgängern modular aufgebaut. 
Dies bedeutet, dass einzelne Teiltechniken so wie die Steuerung von Sprachausgaben eigene
Versionsschritte und Entwicklungszeiten besitzen. Manche Module wurden bereits als fertig empfohlen und werden auch 
schon in Darstellungsprogrammen implementiert.
~\cite{CSS}