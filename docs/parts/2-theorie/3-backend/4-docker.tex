\section{Docker}
\subsection{Was ist Docker?}
Der Begriff \textit{Docker} bezieht sich auf mehrere Dinge. Ein Docker ist sowohl ein Open Source
Community Projekt, als auch das Unternehmen \textit{Docker Inc}, dass das Projekt und seine Tools
hauptsächlich unterstützt.

Unter der IT-Software Docker versteht man also eine Containertechnologie, die die Erstellung von Linux
Containern ermöglicht. Die Container sind in der Regel leichtgewichtige, modulare virtuelle Maschinen
. Dadurch entsteht vor allem Flexibilität, da die Container einfach erstellt, eingesetzt, konzipiert
und zwischen Umgebungen bewegt werden können.

\subsection{Wie genau funktioniert Docker?}
Damit Docker Prozesse unabhängig voneinander ausführen kann, verwendet die Docker Technologie
den Linux Kernel und seine Funktionen wie \textbf{Cgroups} \textbf{namespaces}. Somit werden die
einzelnen Container voneinander isoliert und es ist möglich mehrere Prozesse und Apps getrennt
voneinander ausführen zu können. Dadurch entsteht eine bessere Infrastruktur die gleichzeitig
Sicherheit bewahrt.

Die Containertools arbeiten mit einem Image-basierten Bereitstellungsmodell. Aufgrund dessen ist es
einfach Anwendungen oder Pakete mit all ihren Abhängigkeiten in mehreren Systemen gemeinsam zu benutzen.
Alle diese Tools bauen auf Linux Container auf was ihre Benutzung einfach und einzigartig macht.
Außerdem ermöglicht es eine schnellere Bereitstellung und Kontrolle von Versionen und ihrer Verbreitung.

\subsection{Unterschied Docker Container und herkömmliche Linux-Container}
Ursprünglich basierte die Docker Technologie auf der \textbf{LXC- Technologie}, die mit den
traditionellen Linux Container assoziiert wird. Docker hat sich jedoch mittlerweile von dieser
Abhängigkeit befreit, da LXC keine gute Entwickler und Nutzererfahrung bot. Docker bietet wie bereits
erwähnt mehr Funktionen an als nur die reine Ausführung von Container, sie vereinfacht den Prozess der
Erstellung und des Aufbaus von Container sowie den Versand von Images.

Linux Container verwenden ein init-System, das mehrere Prozesse verwalten kann. Somit können
komplette Anwendungen als eines betrieben werden. Docker hingegen schlüsselt die Anwendungen und
ihre einzelnen Prozesse auf und stellt somit die Tools bereit.

\subsection{Vorteile von Docker}
Ein großer Vorteil von Docker ist, dass bei einer Reparatur oder
Aktualisierung nur ein Teil der Anwendung außer Betrieb genommen werden muss und
der Rest weiter laufen kann. Außerdem ermöglicht es Docker Prozesse in mehreren
Apps gemeinsam zu verwenden.

\subsubsection{Layer und Image}
Ein Docker-Image besteht aus mehreren Layer. Mehrere Layer werden dann
in einem einzelnen Image vereint. Falls ein User einen Befehl wie \textit{copy}
oder \textit{run}, wird ein neuer Layer erstellt.

Diese Layer verwendet Docker dann für die Erstellung neuer Container, was eine enorm schnelle
Entwicklung bedeutet. Bei jedem Image hat man außerdem ein eingebautes Änderungsprotokoll
und somit volle Kontrolle über Container-Images.

\subsubsection{Rollback Funktion?}
Eines der größten Vorteile vom Layering Prinzip ist, das man auf vorherige Versionen
Zurücksetzten kann. Aufgrund diesen Ansatz wird für eine kontinuierliche Integration und
Bereitstellung gesorgt. \textbf{Continuous Integration/Continuous Development CI/CD}

\subsubsection{Schnelle Bereitstellung}
Die neue Bereitstellung von Hardware dauert in der Regel Tage. Mithilfe der Docker-basierten
Container wird die Bereitstellung in Sekunden erledigt. Da für jeden Prozess ein neuer Container
erstellt wird kann man einfach ähnliche Prozesse in Sekunden teilen. Aufgrund das beim Hinzufügen
oder Verschieben eines Containers das Betriebssystem nicht neu gebootet werden muss sind die
Bereitstellungszeiten wesentlich kürzer.

\cite{Docker}
\label{Docker}