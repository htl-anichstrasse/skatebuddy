\section{Datenbank}
\subsection{Definition}
Eine Datenbank ist eine große Sammlung von strukturierten und organisierten Daten,
die in einem Computersystem gespeichert sind. Gesteuert wird eine solche Datenbank von
einem \textbf{Datenbankmanagementsystem (DBMS)}. Die Daten und das DBMS werden zusammen als
Datenbanksystem bezeichnet, jedoch meistens als Datenbank abgekürzt.

Die Daten in den gängigsten verwendeten Datenbanken werden in Zeilen und Spalten unterteilt und
in einer Tabelle modelliert. Aufgrund dieser Struktur können diese Daten dann einfach, abgerufen,
modifiziert, aktualisiert, kontrolliert und organisiert werden. Fast alle Datenbanken verwenden
zum Abfragen oder Schreiben von die \textit{Structured Query Language (SQL)}.
\cite{Datenbank}
\subsection{Structured Query Language (SQL)}
Die Programmiersprache SQL wird heutzutage von fast allen relationalen Datenbanken verwendet
um Daten Abzufragen, Manipulieren und zu Definieren. SQL hat außerdem zu vielen Erweiterungen
bei großen Unternehmen wie IBM, Oracle und Microsoft.

\subsection{MySQL}
\textit{MYSQL} ist ein relationales Datenbankmanagementsystem, das auf dem Open-Source
Programmiersprache \textit{SQL} basiert. Eingeführt wurde es 1995 vom schwedischen Unternehmen
\textit{MySQLAB} und wurde dann im Jahre 2010 von \textit{Oracle} übernommen. Der Name fügt sich
aus der Kombination "My", der Name einer Tochter eines Mitbegründers und ``SQL'' der Abkürzung
für Structured Query Language zusammen.

Um das Prinzip von \textit{MYSQL} zu verstehen muss man zwei Konzepte verstehen.

\subsubsection{Relationale Datenbanken}
Bei Relationalen Datenbanken werden die Daten in Form von Tabellen dargestellt, anstelle von
einem großen Speicherbereich.
Bei relativen Datenbanken gibt es einen sogenannten \textbf{Schlüssel}. Mithilfe diesen Schlüssels
ist es möglich, Daten zwischen verschiedenen Tabellen miteinander zu verknüpfen. In der Regel ist
ein solcher Schlüssel eine \textbf{eindeutige Identifikationsnummer (ID)}.
\subsubsection{Client-Server-Model}
\textit{MySql} basiert

\cite{MySQL}
\label{db}

