\section{Datenbank}
\subsection{Definition}
Eine Datenbank ist eine große Sammlung von strukturierten und organisierten Daten,
die in einem Computersystem gespeichert sind. Gesteuert wird eine solche Datenbank von
einem \textbf{Datenbankmanagementsystem (DBMS)}. Die Daten und das DBMS werden zusammen als
Datenbanksystem bezeichnet, jedoch meistens als Datenbank abgekürzt.

Die Daten in den gängigsten verwendeten Datenbanken werden in Zeilen und Spalten unterteilt und
in einer Tabelle modelliert. Aufgrund dieser Struktur können diese Daten dann einfach, abgerufen,
modifiziert, aktualisiert, kontrolliert und organisiert werden. Fast alle Datenbanken verwenden
zum Abfragen oder Schreiben die \textit{Structured Query Language (SQL)} Sprache.
\cite{Datenbank}

\subsection{Datenbankmanagementsystem}
Als \textit{Datenbankmanagementsystem(DBMS)} wird ein Datenbankprogramm bezeichnet die als Schnittstelle
zwischen Usern und der Datenbank dient. Es ermöglicht den Benutzern
Informationen aufzurufen zu aktualisieren oder zu optimieren. Administrative Aufgaben
werden durch so ein \textit{DBMS} ebenso erleichtert. Abläufe wie Leistungsüberwachung,
Abstimmung sowie Backup und Wiederherstellung werden ermöglicht.

Die wohl bekanntesten \textit{DBMSs} sind:

\begin{itemize}
    \item  MySQL
    \item Microsoft Access
    \item Microsoft SQL Server
    \item File Maker Pro
    \item Oracle database
    \item dBase
\end{itemize}

\subsection{Structured Query Language (SQL)}
Die Programmiersprache SQL wird heutzutage von fast allen relationalen Datenbanken verwendet,
um Daten Abzufragen, Manipulieren und zu Definieren. SQL hat außerdem zu vielen Erweiterungen
bei großen Unternehmen wie IBM, Oracle und Microsoft geführt.

\subsection{MySQL}

\textit{MYSQL} ist ein relationales Datenbankmanagementsystem, das auf der Open-Source
Programmiersprache \textit{SQL} basiert. Eingeführt wurde es 1995 vom schwedischen Unternehmen
\textit{MySQLAB} und wurde dann im Jahre 2010 von \textit{Oracle} übernommen. Der Name fügt sich
aus der Kombination "My", der Name einer Tochter eines Mitbegründers und ``SQL'' der Abkürzung
für Structured Query Language zusammen.

Um das Prinzip von \textit{MYSQL} zu verstehen muss man zwei Konzepte verstehen.

\subsubsection{Relationale Datenbanken}
Bei Relationalen Datenbanken werden die Daten anstelle von
einem großen Speicherbereich in Form von Tabellen dargestellt.
Bei relativen Datenbanken gibt es einen sogenannten \textbf{Schlüssel}. Mithilfe diesen Schlüssels
ist es möglich, Daten zwischen verschiedenen Tabellen miteinander zu verknüpfen. In der Regel ist
ein solcher Schlüssel eine \textbf{eindeutige Identifikationsnummer (ID)}.
\subsubsection{Client-Server-Model}
\textit{MySql} basiert außerdem auf dem Prinzip des \textbf{Client-Server-Model}. Im \textbf{Server}
befinden sich die Daten und um auf diese zugreifen zu können wird ein sogenannter \textbf{Client}
benötigt. Unter \textit{SQL} sendet also der Client eine Anfrage auch \textbf{request} genannt
worauf der Datenbankserver mit den benötigten Daten antwortet \textbf{response} .

Wir haben uns für eine \textit{MySQL} Datenbank entschieden, da wir uns schon in der Schule
damit auseinander gesetzt haben und wir somit schon Erfahrung hatten.

\begin{center}[h]
    \includegraphics[width=0.75\textwidth]{Backend/ClientSever.png}
\end{center}



\subsection{Andere Datenbanktypen}
Wie bereits oben beschrieben, ist \textit{MySql} ein relatives Datenbanksystem. Es gibt
jedoch einige weiter Datenbanktypen.

\subsubsection{Objektorientierte Datenbanken}
Die Informationen werden in einer in Form von Objekten dargestellt.

\subsubsection{Verteilte Datenbanken}
In einer verteilten Datenbank werden die Informationen in zwei oder mehreren Dateien gespeichert.
Diese Dateien können auf mehreren Rechnern verteilt sein, die entweder an einem einzigen Standort
sind oder welche über ein Netzwerk miteinander kommunizieren.

\subsection{Data Warehouse}
Data Warehouse Datenbanken bestehen aus einem zentralen Daten Repository. Es wurde speziell für
schnelle Anfragen und Analysen konzipiert.

\subsection{NoSQL-Datenbanken}
Im Gegensatz zu einem relativen Datenbanksystem ermöglicht eine \textit{NoSql} Datenbank,
semistrukturierter Daten bzw. unstrukturierte Daten zu speichern und verändern. NoSql Datenbanken
setzten zur Organisation nicht auf sogenannte \textbf{keys}, sondern auf Wertepaare, Objekte, Dokumente,
Listen oder Reihen. Da \textit{NoSql} Systeme sehr flexibel sind, eignen sie sich
besonders für große Datenmengen sogenannte \textbf{Big Data}. Die bekanntesten
\textit{NoSql} Datenbanksysteme sind
\begin{itemize}
    \item Apache
    \item Cassandra
    \item Riak
    \item MongoDB
    \item Redis
    \item CouchDB
\end{itemize}

\subsection{Diagrammdatenbanken}
Daten werden anhand von Entitäten und ihren Beziehungen abgespeichert.

\subsection{OLTP-Datenbanken}
Eine \textit{OLTP} Datenbank kennzeichnet sich dadurch, dass sie schnell, analytisch und für eine
Vielzahl von Transaktionen mehrerer Benutzer ausgelegt ist.

Das sind natürlich nicht alle vorhandenen Datenbanktypen. Die Anderen sind jedoch
nicht weit verbreitet und speziell für wissenschaftliche, finanzielle oder andere Funktionen konzipiert.

\subsubsection{Password-Hashing}
Damit in der Datenbank die Passwörter nicht im Klartext angezeigt werden.
Müssen sie vorher \textbf{gehashed} werden.

\paragraph{Was bedeutet Hashing?}
Das sogenannte Password-Hashing dient einem erhöhten Datenschutz. Er
wandelt das Klartext Passwort in eine Reihe von Zeichen und Symbolen-folgen an.

\paragraph{Wie funktioniert das Hashing?}
Mittels eines Password Hashing Verfahrens werden Passwörter in einer festgelegten
Codefolge in zufälligen Buchstaben und Zahlen umgewandelt. Die Generierung der
sogenannten \textit{Hashes} läuft also automatisiert mit einem Hash-Algorithmus ab.
Die bekanntesten Hashing Algorithmen sind \textbf{Skrypt} oder \textbf{Argon2}.

\paragraph{Salted Passwords}
Da ein Algorithmus gleiche Passwörter immer gleich Hashed und man somit leicht
das Klartext Passwort raus filtern kann wird der Hash noch "gesalzen". Ein \textit{Salt}
ist eine zufällig generierte Ziffer die in dem Hash einfließt. Somit kann also verhindert
werden das der gleiche Hash öfter in der Datenbank steht.
\cite{Hashing}

\begin{code}[H]
    \begin{lstlisting}[firstnumber=1,language=JavaScript, style=CMD]
        router.post('/register', async (req, res, next) => {
            const password = req.body.password;
            const encryptedPassword = await bcrypt.hash(password, saltRounds);
            try {
                const temp = new User(
                    null,
                    req.body.name,
                    encryptedPassword,
                    req.body.email,
                );
                var alreadyExists = await User.alreadyExists(
                    con,
                    temp.name,
                    temp.email,
                );
                if (!alreadyExists) {
                    await User.insertValue(con, temp);
                    const user = await User.getByEmail(con, temp.email);
                    token = User.generateToken(user);
                    res.send({ success: true, token: token });
                } else {
                    res.send({ success: false, message: 'User already exists!' });
                }
            } catch (e) {
                console.log(e);
                res.sendStatus(500);
            }
        });
    \end{lstlisting}
    \caption{Code-Snippet-Hashing}
\end{code}

\paragraph{Erklärung zum Code:}
In Zeile zwei wird das Klartext Passwort aus dem Body entnommen. Des weiteren
wird in der nächsten Zeile das Klartext Passwort mittels der asynchronen Methode
\texttt{bcrypt.hash(klartext Passwort, saltRounds)}. Die saltRounds sind ein
Faktor der bestimmt wie lange es dauert um ein einziges BCrypt Hash zu erstellen.
Den Faktor um 1 zu erhöhen würde eine doppelt so lange Hash Zeit bedeuten.
Im Weiterfolgeanden Code wird zuerst gecheckt ob ein User mit der Email oder dem Username bereits
besteht. Falls der Username und die Email unique sind wird der Benutzer
in die Datenbank eingetragen und ein Token \underline{\nameref*{restapi}} wird erstellt.
\cite{MySQL}
\label{db}

