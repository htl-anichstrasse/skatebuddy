\subsection{Node.js}
\label{nodejs}

\textit{Node.js} ist eine serverseitige, eventbasierte Laufzeitumgebung für die
Programmiersprache \textit{JavaScript},
mit der es möglich ist \textit{JavaScript} Code
außerhalb des Webbrowsers auszuführen.

Sie wurde 2009 von Ryan Dahl entwickelt als Anwendung, für die Entwicklung skalierbarer
Netzwerkanwendungen.

NodeJs basiert auf der V8-Engine also auf der selben Engine wie \textit{Google Chrome}.
Das heißt also, dass NodeJs die selbe Engine zum Ausführen von JavaScript-Code verwendet.
wie \textit{Google Chrome}

Obwohl NodeJS hauptsächlich für die Programmierung von, Webserver
Programme eingesetzt wird, kann es auch für die Entwicklung von
Skripten oder Tools oder bei Entwicklung von Desktop- oder Echtzeitanwendungen
verwendet werden.

\subsubsection{Vorteile von NodeJs}
Einer der größten Vorteile von NodeJS gegenüber anderen Web-Serving-Techniken ist,
dass es sich bei der Kommunikation um eine \textbf{Non-Blocking I/0} also Einer
\textbf{asynchronen } Kommunikation handelt.

Außerdem wird keine zusätzlicher Server benötigt bei einer NodeJS Anwendung,
da die Anwendung auch den Webserver darstellt. Somit haben Server und Webserver
die selbe Programmiersprache \textbf{Javascript} was wiederum ein großer Vorteil ist.

NodeJs beinhaltet außerdem eine Menge \textbf{Build-in-Modulen}, ohne Funktionen
die ohne zusätzliche Installation zur Verfügung stehen.
Zur Installation von zusätzlichen
\cite{NodeJs}

