\subsection{Node.js}
\label{nodejs}

\textit{Node.js} ist eine serverseitige, eventbasierte Laufzeitumgebung für die
Programmiersprache \textit{JavaScript} (\nameref{js}),
mit der es möglich ist \textit{JavaScript} Code
außerhalb des Webbrowsers auszuführen.

Sie wurde 2009 von Ryan Dahl entwickelt als Anwendung, für die Entwicklung skalierbarer
Netzwerkanwendungen.

NodeJs basiert auf der \textit{V8-Engine} also auf der selben Engine wie \textit{Google Chrome}.
Das heißt also, dass NodeJs die selbe Engine zum Ausführen von JavaScript-Code verwendet
wie \textit{Google Chrome}.
Die \textit{V8-Engine} ist eine prozessbasierte virtuelle Maschine, die
mittels eines Just-In-Time (JIT) Compilers, den geschriebenen JavaScript Code,
zur Laufzeit in Maschinencode übersetzt.

Obwohl \textit{NodeJS} hauptsächlich für die Programmierung von, Webserver
Programme eingesetzt wird, kann es auch für die Entwicklung von
Skripten oder Tools oder bei Entwicklung von Desktop- oder Echtzeitanwendungen
verwendet werden.

\subsubsection{Vorteile von NodeJs}
Einer der größten Vorteile von \textit{NodeJS} gegenüber anderen Web-Serving-Techniken ist,
dass es sich bei der Kommunikation um eine \textbf{Non-Blocking I/0} also Einer
\textbf{asynchronen } Kommunikation handelt.

Außerdem wird keine zusätzlicher Server bei einer NodeJS Anwendung benötigt,
da die Anwendung auch den Webserver darstellt. Somit haben Server und Webserver
die selbe Programmiersprache \textit{Javascript} was wiederum ein großer Vorteil ist.

NodeJs beinhaltet außerdem eine Menge \textbf{Build-in-Module},
die ohne zusätzliche Installation zur Verfügung stehen.
Zwecks Installation, Entfernung und Aktualisierung von zusätzlichen Packages verfügt Node über den
sogenannten \textit{Node Package Manager (npm)}. Alternativ kann auch der Packetmanager
\textit{Yarn} verwendet werden.

Packages sind externe Bibliotheken die untergeordnete Aufgaben in Projekten durchführen.


\subsection{NodeJs Anwendungsbereich}
Umfragen des Codings-Forums \textit{Stack Overflow} ergab das rund 50.4\%
der professionellen Entwickler NodeJs als Backend Framework verwenden.

Große Unternehmen wie:
\begin{itemize}
    \item Netflix
    \item Paypal
    \item Uber
    \item Linkedin
\end{itemize}
verwenden ebenfalls das Framework.
\cite{NodeJs1}
\cite{NodeJs2}

