\section{React Native: Einmal lernen, überall schreiben}
\label{reactnative}

\subsection{Was ist React Native?}
React Native ist ein JavaScript-Framework, welches dafür verwendet wird um Cross-Platform-Apps für
die Betriebssysteme Android, iOS, Windows und Web zu entwickeln.
Außerdem ist der Zugriff auf die native Plattform-API möglich.

\subsection{Wer benutzt React Native?}
Laut der eigenen Webseite von React Native benutzen einige der größten Firmen der Welt React Native
als Cross-Platform-Framework für ihre Android- und iOS-Apps.

\begin{itemize}
  \item[\Square] offener Punkt
  \item[\XBox] angekreuzter Punkt
  \item[\CheckedBox] abgehakter Punkt
\end{itemize}

\cite{reactNativeShowcase}

\subsection{Geschichte}

\subsection{Lizenz}

\subsection{Das React in React Native}
Wie der Name schon vermuten lässt baut React Native hauptsächlich auf die JavaScript-Bibliothek
React auf. Sie hilft bei der Erstellung der Benutzeroberfläche und verwaltet den Zustand der
Anwendung.

Detaillierte Informationen zu State, Render-Funktionen, Komponenten und JSX finden Sie im Kapitel
\nameref{reactjs}.

\subsection{Benutzeroberfläche}

\subsubsection{Core Components}


\section{Expo CLI und Expo Go}
Die Entwickler von React Native selbst empfehlen allen Anfängern im Gebiet App-Entwicklung die
Verwendung der Expo CLI, welche eine vereinfachte 

Die simpelsten Anwendungen können theoretisch auf jedem Zielbetriebssystem ausgeführt werden, also
Android, iOS, Windows und Web. Sollte man in seiner Anwendung Zugriff auf gerätespezifische APIs
oder Konfigurationsdateien benötigen.

\section{React Native CLI}