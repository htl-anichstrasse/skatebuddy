\section{React Native}
\label{reactnative}

\subsection{Was ist React Native?}
React Native ist ein JavaScript-Framework, welches dafür verwendet wird, Cross-Platform-Apps für
die Betriebssysteme Android, iOS, Windows und MacOS zu entwickeln. Außerdem ist der Zugriff auf die
native Plattform-API möglich.

\subsection{Wie viel React steckt wirklich in React Native?}
Wie der Name schon vermuten lässt baut React Native hauptsächlich auf die JavaScript-Bibliothek
React auf. Sie hilft bei der Erstellung der Benutzeroberfläche, verwaltet den Zustand der
Anwendung und hat noch viele weitere kleine Aufgaben. React an sich ist noch eine Bibliothek,
mit -- absichtlich -- wenig "Grundgerüst". Das macht es auch für die Open-Source-Gemeinschaft
einfacher zusammen gute Lösungen zu entwickeln und so werden die wenigen wichtigen Aufgaben von
React dafür perfekt ausgeführt.
Detaillierte Informationen zu State, Render-Funktionen, Komponenten und JSX finden Sie im Kapitel
\nameref{reactjs}.

\subsection{Geschichte}
Die ersten Versionen der Facebook-App für Smartphones war eine Hybrid-App, also eine Webseite,
welche einfach in einer nativen App eingebunden wird \cite{reactNativeHistory}. Die Leistung der
Handy-App war für das Niveau des Social-Media-Giganten eindeutig zu schlecht, die App musste neu
geschrieben werden in Objective-C, womit Facebook sich eine 2,5-fach schnellere Anwendung erwartete
\cite{facebookNewIosApp}.

Nachdem React veröffentlicht worden war, versuchten die Entwickler bei Facebook einen effizienten
Weg zu finden, die neue Technologie zu nutzen um native Anwendungen, genauer eigentlich
Cross-Platform Anwendungen, zu schreiben.

2015 wurde die erste stabile Version von React Native veröffentlicht. Seit Anfang an verwendet
Facebook intern in ihren erfolgreichsten Apps React Native als Basistechnologie. Die Firma Microsoft
stellt selbst Bibliotheken zur Verfügung, um React Native Anwendungen auch für die Universal Windows
Platform (UWP) und MacOS zu entwickeln.

\newpage

\subsection{Wer benutzt React Native?}
Laut der eigenen Webseite von React Native benutzen einige der größten Firmen der Welt React Native
als Cross-Platform-Framework für ihre Android- und iOS-Apps.

\begin{table}[H]
\centering
\begin{tabular}{|l|l|c|c|l|}
  \hline
  \textbf{Unternehmen} & \textbf{App} & \textbf{Android} & \multicolumn{1}{l|}{\textbf{iOS}} \\ \hline\hline
  \multirow{5}{*}{Facebook} & Facebook             & \multicolumn{2}{c|}{\multirow{5}{*}{\XBox}} \\
                            & Facebook Ads Manager & \multicolumn{2}{c|}{}                       \\
                            & Facebook Analytics   & \multicolumn{2}{c|}{}                       \\
                            & Instagram            & \multicolumn{2}{c|}{}                       \\
                            & Oculus               & \multicolumn{2}{c|}{}                       \\ \hline
  Microsoft                 & Skype                & \multicolumn{2}{c|}{\XBox}                  \\ \hline
  Discord                   & Discord              & \Square          & \XBox                    \\ \hline
  Tesla                     & Tesla                & \multicolumn{2}{c|}{\XBox}                  \\ \hline
  Coinbase                  & Coinbase             & \multicolumn{2}{c|}{\XBox}                  \\ \hline
  Walmart                   & Walmart              & \multicolumn{2}{c|}{\XBox}                  \\ \hline
  Pinterest                 & Pinterest            & \multicolumn{2}{c|}{\XBox}                  \\ \hline
  Uber                      & Uber Eats            & \multicolumn{2}{c|}{\XBox}                  \\ \hline
  Shopify                   & Shopify              & \multicolumn{2}{c|}{\XBox}                  \\ \hline
  Wix.com                   & Wix.com              & \multicolumn{2}{c|}{\XBox}                  \\ \hline
\end{tabular}
\end{table}

\begin{center}
  Berühmte Beispiele für Apps mit React Native \cite{reactNativeShowcase}
\end{center}

Discord ist das einzige Unternehmen in der Liste, welches nur für die Plattform iOS React Native
verwendet. In einem Blog Post erklären sie, dass React Native auf Android ihrer Meinung nach keine
akzeptable Performance in Sachen Reaktionsfähigkeit von Knöpfen liefert \cite{reactNativeDiscord}.
Dieser Post ist aus dem Jahr 2018 und seitdem hat sich React Native stark verbessert. Im selben Jahr
begann nämlich auch die Entwicklung von Fabric, der neuen Render-Engine von React Native, welche
2021 dann schließlich in die offizielle Facebook App übernommen wurde.

\subsection{Lizenz}
React Native wird unter der MIT-Lizenz geführt, das heißt dass jeder die Software sowohl für
Open-Source als auch für Closed-Source-Projekte verwenden darf.
