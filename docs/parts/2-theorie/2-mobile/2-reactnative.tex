\section{React Native: Einmal lernen, überall schreiben}
\label{reactnative}

\subsection{Was ist React Native?}
React Native ist ein JavaScript-Framework, welches dafür verwendet wird um Cross-Platform-Apps für
die Betriebssysteme Android, iOS, Windows und Web zu entwickeln. Außerdem ist der Zugriff auf die
native Plattform-API möglich.

\subsection{Geschichte}
Die ersten Versionen der Facebook-App für Smartphones war eine Hybrid-App, also eine Webseite,
welche einfach in einer nativen App eingebunden wird \cite{reactNativeHistory}. Die Leistung der
Handy-App war für das Niveau des Social-Media-Giganten eindeutig zu schlecht, die App musste neu
geschrieben werden in Objective-C, womit Facebook sich eine 2,5-fach schnellere Anwendung erwartete
\cite{facebookNewIosApp}.

Nachdem React veröffentlicht worden war, versuchten die Entwickler bei Facebook einen effizienten
Weg zu finden, die neue Technologie zu nutzen um native Anwendungen, genauer eigentlich
Cross-Platform Anwendungen, zu schreiben.

2015 wurde die erste stabile Version von React Native veröffentlicht. Seit Anfang an verwendet
Facebook intern in ihren erfolgreichsten Apps React Native als Basistechnologie. Die Firma Microsoft
stellt selbst Bibliotheken zur Verfügung, um React Native Anwendungen auch für die Universal Windows
Platform (UWP) und MacOS zu entwickeln.

\subsection{Wer benutzt React Native?}
Laut der eigenen Webseite von React Native benutzen einige der größten Firmen der Welt React Native
als Cross-Platform-Framework für ihre Android- und iOS-Apps.

\begin{center}
\begin{table}[H]
  \begin{tabular}{lllll}
    \hline
    \multicolumn{1}{|l|}{\textbf{Unternehmen}}       & \multicolumn{1}{l|}{\textbf{App}}         & \multicolumn{1}{l|}{\textbf{Android}} & \multicolumn{1}{l|}{\textbf{iOS}} & \multicolumn{1}{l|}{\textbf{Bemerkung}} \\ \hline
    \multicolumn{1}{|l|}{\multirow{5}{*}{Facebook}}  & \multicolumn{1}{l|}{Facebook}             & \multicolumn{2}{c|}{\multirow{5}{*}{\XBox}}                               & \multicolumn{1}{l|}{}                   \\
    \multicolumn{1}{|l|}{}                           & \multicolumn{1}{l|}{Facebook Ads Manager} & \multicolumn{2}{c|}{}                                                     & \multicolumn{1}{l|}{}                   \\
    \multicolumn{1}{|l|}{}                           & \multicolumn{1}{l|}{Facebook Analytics}   & \multicolumn{2}{c|}{}                                                     & \multicolumn{1}{l|}{}                   \\
    \multicolumn{1}{|l|}{}                           & \multicolumn{1}{l|}{Instagram}            & \multicolumn{2}{c|}{}                                                     & \multicolumn{1}{l|}{}                   \\
    \multicolumn{1}{|l|}{}                           & \multicolumn{1}{l|}{Oculus}               & \multicolumn{2}{c|}{}                                                     & \multicolumn{1}{l|}{}                   \\ \hline
    \multicolumn{1}{|l|}{\multirow{1}{*}{Coinbase}}  & \multicolumn{1}{l|}{Coinbase}             & \multicolumn{2}{c|}{\XBox}                                                & \multicolumn{1}{l|}{}                   \\ \hline
    \multicolumn{1}{|l|}{\multirow{1}{*}{Shopify}}   & \multicolumn{1}{l|}{Shopify}              & \multicolumn{2}{c|}{\XBox}                                                & \multicolumn{1}{l|}{}                   \\ \hline
    \multicolumn{1}{|l|}{\multirow{1}{*}{Discord}}   & \multicolumn{1}{l|}{Discord}              & \multicolumn{1}{c|}{\Square}          & \multicolumn{1}{c|}{\XBox}        & \multicolumn{1}{l|}{}                   \\ \hline
    \multicolumn{1}{|l|}{\multirow{1}{*}{Microsoft}} & \multicolumn{1}{l|}{Skype}                & \multicolumn{2}{c|}{\XBox}                                                & \multicolumn{1}{l|}{}                   \\ \hline
    \multicolumn{1}{|l|}{\multirow{1}{*}{Pinterest}} & \multicolumn{1}{l|}{Pinterest}            & \multicolumn{2}{c|}{\XBox}                                                & \multicolumn{1}{l|}{}                   \\ \hline
    \multicolumn{1}{|l|}{\multirow{1}{*}{Tesla}}     & \multicolumn{1}{l|}{Tesla}                & \multicolumn{2}{c|}{\XBox}                                                & \multicolumn{1}{l|}{}                   \\ \hline
    \multicolumn{1}{|l|}{\multirow{1}{*}{Uber}}      & \multicolumn{1}{l|}{Uber Eats}            & \multicolumn{2}{c|}{\XBox}                                                & \multicolumn{1}{l|}{}                   \\ \hline
    \multicolumn{1}{|l|}{\multirow{1}{*}{Walmart}}   & \multicolumn{1}{l|}{Walmart}              & \multicolumn{2}{c|}{\XBox}                                                & \multicolumn{1}{l|}{}                   \\ \hline
    \multicolumn{1}{|l|}{\multirow{1}{*}{Wix.com}}   & \multicolumn{1}{l|}{Wix.com}              & \multicolumn{2}{c|}{\XBox}                                                & \multicolumn{1}{l|}{}                   \\ \hline
  \end{tabular}
\end{table}

Berühmte Beispiele für Apps mit React Native \cite{reactNativeShowcase}
\end{center}

\subsection{Lizenz}
React Native wird unter der MIT-Lizenz geführt, das heißt dass jeder die Software sowohl für
Open-Source als auch für Closed-Source-Projekte verwenden darf.

\subsection{Das React in React Native}
Wie der Name schon vermuten lässt baut React Native hauptsächlich auf die JavaScript-Bibliothek
React auf. Sie hilft bei der Erstellung der Benutzeroberfläche, verwaltet den Zustand der
Anwendung und hat noch viele weitere kleine Aufgaben. React an sich ist noch eine Bibliothek,
mit -- absichtlich -- wenig "Grundgerüst". Das macht es auch für die Open-Source-Gemeinschaft
einfacher zusammen gute Lösungen zu entwickeln und so werden die wenigen wichtigen Aufgaben von
React dafür perfekt ausgeführt.

Detaillierte Informationen zu State, Render-Funktionen, Komponenten und JSX finden Sie im Kapitel
\nameref{reactjs}.

\subsection{Benutzeroberfläche}


\subsubsection{Core Components}


\section{Expo CLI und Expo Go}
Die Entwickler von React Native selbst empfehlen allen Anfängern im Gebiet App-Entwicklung die
Verwendung der Expo CLI, welche eine vereinfachte 

Die simpelsten Anwendungen können theoretisch auf jedem Zielbetriebssystem ausgeführt werden, also
Android, iOS, Windows und Web. Sollte man in seiner Anwendung Zugriff auf gerätespezifische APIs
oder Konfigurationsdateien benötigen.

\section{React Native CLI}