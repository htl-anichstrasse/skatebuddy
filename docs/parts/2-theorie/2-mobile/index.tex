\chapter{Mobile App Entwicklung}
\label{mobiledev}

\section{Apps}
Als Apps (von engl. Applications), zu deutsch Anwendungen, werden Softwarepakete für Mobilgeräte
bezeichnet. Es gibt verschiedenste App-Genres, die alle ihren eigenen Zweck erfüllen.
Von einfachen ToDo-Listen-Apps bis zu komplexen Buchhaltungssystemen kann praktisch alles realisiert
werden, wofür die Leistung des Geräts reicht.

Apps können durch mehrere Herangehensweisen entwickelt werden.

\subsection{Native Apps}
Die Applikation wird direkt auf das Zielbetriebssystem aufgebaut und hat tiefen Zugriff auf die
Software- und Hardwareschnittstellen des Geräts.

\subsubsection{Vorteil}
Ein großer Vorteil davon ist die Performance, es gibt keine Zwischen-Ebene, mit der kommuniziert
werden muss.

\subsubsection{Nachteil}
Die meisten Firmen wollen ihre Anwendungen auf beiden Plattformen anbieten, also sowohl eine
Android-App im Google Play Store, als auch eine iOS-App im App Store. Um beides mit Nativen Apps
zu realisieren benötigt man also zwei App-Entwicklungs-Teams, die zwei verschiedene Repositories
pflegen müssen und welche beide gleichzeitig neue Features einbauen und testen müssen.

\subsection{Webanwendungen}
Eine mögliche Lösung ist die Webanwendung. Alles, was dafür auf dem Endgerät vorhanden sein muss,
ist ein Webbrowser, der die Webseite darstellen kann. Diese Webseite muss nun nur noch von einem
einzigen Team entwickelt und an unterschiedliche Bildschirmgrößen angepasst werden.

Durch die Verwendung einer Web-App im Browser geht jedoch ein großer Teil des Zugriffs auf die
gerätespezifische API verloren. Außerdem funktioniert eine Webanwendung nur mit einer aktiven
Internetverbindung und es muss bei jedem Screen-Wechsel ein neuer Web-Request verarbeitet werden.

\subsection{Hybrid-Apps}
In einer Hybrid-App werden Native App und Webanwendung kombiniert. Die Native App zeigt in einer
sogenannten WebView die gewünschte Webseite an und hat gleichzeitig auch Zugriff auf Software- und
Hardware-Schnittstellen. Der Endbenutzer sollte dabei nicht mitbekommen, dass die Anwendung
eigentlich nur einen Webbrowser darstellt.

\subsection{Cross-Platform-Apps}
Eine Cross-Platform-App hat genau wie eine Hybrid-App Zugriff auf tiefere Ebenen der Geräte-API,
der Unterschied besteht nur darin, dass die Benutzeroberfläche von einer einheitlichen
Auszeichnungssprache (Markup Language) wie XML bzw. XAML oder JSX in die jeweiligen UI-Elemente der
Zielplattform umgewandelt werden.

\newpage
\section{Android: Entwicklung für die OpenSource-Welt}
\label{androiddev}

\subsection{Was ist Android?}
Das Betriebssystem \textit{Android} basiert auf dem Linux-Kernel und ist seit 2012 das am meisten
verwendete Betriebssystem für Smartphones mit einem Marktanteil von etwa 70\% weltweit
\cite{mobileOsMarketShare}. Die aktuelle Version ist Android 12, sie wurde am 4. Oktober 2021
veröffentlicht.

\textit{Android} selbst ist freie Software \cite{androidOpenSourceProject}, welche dadurch definiert
ist, dass jeder Anwender die Software für jeden Zweck verwenden, alle Teile des Quellcodes
ändern und das Ergebnis kopieren und verteilen darf. Jedoch werden die meisten Android-Mobilgeräte
mit vorinstallierter, nicht-freier Software von den Geräte-Herstellern ausgestattet, was die
eigentliche Installation auf dem Gerät proprietär macht.

\subsection{Geschichte}
Im Jahre 2008 veröffentlichte \textit{Google LLC} die erste Version des Betriebssystems, welches
zuvor von Andrew Rubin für die Steuerung von Digitalkameras entwickelt wurde. Es sollte in den
nachfolgenden Jahren von der \textit{Open Handset Alliance}, einem Unternehmenszusammenschluss
aus Firmen im IT-Sektor, weiterentwickelt werden und offene Standards für Mobilgeräte festlegen
\cite{androidHistory}. Es werden seitdem jedes Jahr neue Android-Versionen veröffentlicht, die
neueste Version ist Android 12. Frühere Versionen verwendeten Süßigkeiten als Codenamen:

\begin{itemize}
  \item Android 4.1: Jelly Bean
  \item Android 4.4: KitKat
  \item Android 5: Lollipop
  \item Android 6: Marshmallow
  \item Android 7: Nougat
  \item Android 8: Oreo
  \item Android 9: Pie
\end{itemize}

Diese Codenamen steigen alphabetisch, daher wird Android 10, obwohl es keinen echten Codenamen mehr
trägt, auch manchmal mit einem (Q) geschrieben.

\subsection{Verwendung}
\textit{Google} stellt selbst eine Reihe von Mobilgeräten her, darunter bis 2015 das Nexus und
heutzutage das Google Pixel, mit dem Google Pixel 6 Pro als Flaggschiff von 2021. Die Smartphones
sind mit einer Basis-Version von Android ausgestattet, wohingegen die Tablets und Notebooks Chrome
OS verwenden.

Viele Geräte-Hersteller entwickeln und pflegen eine eigene Abwandlung des Android Open Source
Projekts, sogenannte Aufsätze. Dabei werden meistens Elemente der Benutzeroberfläche stark, jedoch
der grundsätzliche Aufbau gar nicht bis minimal verändert. Bekannte Android-Aufsätze sind:

\begin{itemize}
  \item One UI von Samsung (Südkorea)
  \item MIUI von Xiaomi (China)
  \item EMUI von Huawei (China)
  \item OxygenOS von OnePlus (China)
\end{itemize}

\subsection{App-Entwicklung}
Für Android wird von Google die Entwicklungsumgebung Android Studio zur Verfügung gestellt, welche
auf IntelliJ IDEA von Jetbrains aufbaut und alle benötigten Werkzeuge, für das Entwickeln
von Apps bereitstellt. Außerdem werden Microsoft Windows, Apple MacOS und Linux als
Entwicklungsplattform unterstützt.

Die Programmierung der Anwendung erfolgt hauptsächlich mittels Java und XML. Alternativ zu Java kann
auch Kotlin, eine Programmiersprache, welche extra für die Erstellung von Smartphone-Apps optimiert
wurde, benutzt werden.

Das Basiselement für die Entwicklung ist das Android Software Development Kit (SDK), zu deutsch
Android Softwareentwicklungspaket. Dieses kann durch die offizielle Entwicklungsumgebung Android
Studio heruntergeladen und installiert werden.

\newpage
\section{iOS: Entwicklung für das Apple-Ökosystem}
\label{iosdev}

\subsection{App-Entwicklung}
Um Apps für das Apple-Ökosystem zu entwickeln, benötigt man natürlich einen PC, auf welchem MacOS
installiert ist. Die Programmierung erfolgt hauptsächlich in der Hauseigenen IDE von Apple, XCode,
und den Sprachen Objective-C und, seit 2014, auch mittels Swift.
\newpage
\section{React Native}
\label{reactnative}

\subsection{Was ist React Native?}
React Native ist ein JavaScript-Framework, welches dafür verwendet wird, Cross-Platform-Apps für
die Betriebssysteme Android, iOS, Windows und MacOS zu entwickeln. Außerdem ist der Zugriff auf die
native Plattform-API möglich.

\subsection{Wie viel React steckt wirklich in React Native?}
Wie der Name schon vermuten lässt baut React Native hauptsächlich auf die JavaScript-Bibliothek
React auf. Sie hilft bei der Erstellung der Benutzeroberfläche, verwaltet den Zustand der
Anwendung und hat noch viele weitere kleine Aufgaben. React an sich ist noch eine Bibliothek,
mit -- absichtlich -- wenig "Grundgerüst". Das macht es auch für die Open-Source-Gemeinschaft
einfacher zusammen gute Lösungen zu entwickeln und so werden die wenigen wichtigen Aufgaben von
React dafür perfekt ausgeführt.
Detaillierte Informationen zu State, Render-Funktionen, Komponenten und JSX finden Sie im Kapitel
\nameref{reactjs}.

\subsection{Geschichte}
Die ersten Versionen der Facebook-App für Smartphones war eine Hybrid-App, also eine Webseite,
welche einfach in einer nativen App eingebunden wird \cite{reactNativeHistory}. Die Leistung der
Handy-App war für das Niveau des Social-Media-Giganten eindeutig zu schlecht, die App musste neu
geschrieben werden in Objective-C, womit Facebook sich eine 2,5-fach schnellere Anwendung erwartete
\cite{facebookNewIosApp}.

Nachdem React veröffentlicht worden war, versuchten die Entwickler bei Facebook einen effizienten
Weg zu finden, die neue Technologie zu nutzen um native Anwendungen, genauer eigentlich
Cross-Platform Anwendungen, zu schreiben.

2015 wurde die erste stabile Version von React Native veröffentlicht. Seit Anfang an verwendet
Facebook intern in ihren erfolgreichsten Apps React Native als Basistechnologie. Die Firma Microsoft
stellt selbst Bibliotheken zur Verfügung, um React Native Anwendungen auch für die Universal Windows
Platform (UWP) und MacOS zu entwickeln.

\subsection{Wer benutzt React Native?}
Laut der eigenen Webseite von React Native benutzen einige der größten Firmen der Welt React Native
als Cross-Platform-Framework für ihre Android- und iOS-Apps.

\begin{table}[H]
\centering
\begin{tabular}{|l|l|c|c|l|}
  \hline
  \textbf{Unternehmen} & \textbf{App} & \textbf{Android} & \multicolumn{1}{l|}{\textbf{iOS}} & \textbf{Bemerkung} \\ \hline\hline
  \multirow{5}{*}{Facebook} & Facebook             & \multicolumn{2}{c|}{\multirow{5}{*}{\XBox}} & \\
                            & Facebook Ads Manager & \multicolumn{2}{c|}{}                       & \\
                            & Facebook Analytics   & \multicolumn{2}{c|}{}                       & \\
                            & Instagram            & \multicolumn{2}{c|}{}                       & \\
                            & Oculus               & \multicolumn{2}{c|}{}                       & \\ \hline
  Microsoft                 & Skype                & \multicolumn{2}{c|}{\XBox}                  & \\ \hline
  Discord                   & Discord              & \Square          & \XBox                    & \\ \hline
  Tesla                     & Tesla                & \multicolumn{2}{c|}{\XBox}                  & \\ \hline
  Coinbase                  & Coinbase             & \multicolumn{2}{c|}{\XBox}                  & \\ \hline
  Walmart                   & Walmart              & \multicolumn{2}{c|}{\XBox}                  & \\ \hline
  Pinterest                 & Pinterest            & \multicolumn{2}{c|}{\XBox}                  & \\ \hline
  Uber                      & Uber Eats            & \multicolumn{2}{c|}{\XBox}                  & \\ \hline
  Shopify                   & Shopify              & \multicolumn{2}{c|}{\XBox}                  & \\ \hline
  Wix.com                   & Wix.com              & \multicolumn{2}{c|}{\XBox}                  & \\ \hline
\end{tabular}
\end{table}

\begin{center}
  Berühmte Beispiele für Apps mit React Native \cite{reactNativeShowcase}
\end{center}

\subsection{Lizenz}
React Native wird unter der MIT-Lizenz geführt, das heißt dass jeder die Software sowohl für
Open-Source als auch für Closed-Source-Projekte verwenden darf.

\section{Erstellung eines React Native Projekts}
Schon bei der Erstellung des Projekts müssen einige Fragen geklärt werden. Als erstes sollte man
sich wohl fragen, mit welcher CLI man das Projekt erstellen möchte.

Mit CLI (für engl. command-line-interface) ist in unserem Kontext eine Sammlung von Befehlen
gemeint, die in der Kommandozeile eines PCs ausgeführt werden können. Beispiele für Kommandozeilen-
Emulatoren:

\begin{itemize}
  \item Windows Powershell
  \item Windows Cmd
  \item MacOS zsh
  \item Git Bash
  \item Gnome Shell
\end{itemize}

\subsection{Expo CLI}
Die Entwickler von React Native selbst empfehlen allen Anfängern im Gebiet App-Entwicklung die
Verwendung der Expo CLI, welche eine vereinfachte Variante einer React Native Anwendung erzeugt.

Als Vorteil zählt auf jeden Fall die Geschwindigkeit, mit der eine neue App auf einem neuen Gerät
getestet werden kann. Dies ist meist innerhalb weniger Minuten möglich.

Ein wichtiger Nachteil ist jedoch, dass man in einem Expo-Projekt eingeschränkten Zugriff auf
Schnittstellen des Betriebssystems hat, es sind im Projekt nicht einmal die Ordner android und ios
vorhanden, um Änderungen vorzunehmen.

Um eine Expo React Native Anwendung erstellen zu können benötigt man als erstes \nameref{nodejs} und
den darin enthaltenen Node Package Manager. In einer Kommandozeile führt man nun folgende Befehle
aus, um die Expo-CLI im globalen Kontext zu installieren und anschließend ein Expo Projekt zu
erstellen.

\begin{lstlisting}
C:\example> npm install -g expo-cli
added 1549 packages, and audited 1550 packages in 1m

C:\example>expo init expoInitBlank
? Choose a template: - Use arrow-keys. Return to submit.
    ----- Managed workflow -----
>   blank               a minimal app as clean as an empty canvas
    blank (TypeScript)  same as blank but with TypeScript configuration
    tabs (TypeScript)   several example screens and tabs using react-navigation and TypeScript
    ----- Bare workflow -----
    minimal             bare and minimal, just the essentials to get you started

√ Choose a template: » blank               a minimal app as clean as an empty canvas
√ Downloaded template.
🧶 Using Yarn to install packages. Pass --npm to use npm instead.
√ Installed JavaScript dependencies.

✅ Your project is ready!

To run your project, navigate to the directory and run one of the following yarn commands.

- cd expoInitBlank
- yarn start # you can open iOS, Android, or web from here, or run them directly with the commands below.
- yarn android
- yarn ios # requires an iOS device or macOS for access to an iOS simulator
- yarn web
\end{lstlisting}


\subsection{}

Die simpelsten Anwendungen können theoretisch auf jedem Zielbetriebssystem ausgeführt werden, also
Android, iOS, Windows und Web. Sollte man in seiner Anwendung Zugriff auf gerätespezifische APIs
oder Konfigurationsdateien benötigen.

\section{React Native CLI}
npx seit npm 5.2.0
\newpage
\section{Alternative Kandidaten}
\subsection{Flutter}
\label{flutter}
Flutter ist ein Framework, welches von Google erstellt wurde und heutzutage zu den modernsten seiner
Art gehört. Es benutzt für die Benutzeroberflächenerstellung die neuartige Sprache Dart, die extra
für dieses Framework entwickelt wurde und kann für die Erstellung von Cross-Platform-Applikationen
eingesetzt werden. Es unterstützt die Zielplattformen Android, iOS, Windows, MacOS, Linux, Web und
sogar Google Fuchsia \cite{flutter}.

\subsection{Xamarin}
\label{xamarin}
Das Cross-Plattform-Framework Xamarin ist, im Vergleich mit Xamarin und React Native, schon am
Längsten in Entwicklung. Es wird hauptsächlich in den Sprachen C\# und XAML verfasst und ist auch
für praktisch alle Zielplattformen geeignet. Jedoch gibt es einen entscheidenden Nachteil: Für jede
Zielplattform gibt es eine eigene Version des Frameworks. Es teilt sich auf in Xamarin\.Native und
Xamarin.Forms, letzteres unterstützt die Cross-Plattform Eigenschaften.

2020 wurde von Microsoft verkündet, dass sie eine Fusion aus Xamarin und .NET 6 planen, um ein
neues Framework zu erschaffen. Dieses soll den Namen .NET MAUI (.NET Multi-Platform App UI) tragen
\cite{xamarin}.
\newpage