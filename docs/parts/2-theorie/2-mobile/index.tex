\chapter{Mobile App Entwicklung}
\label{mobiledev}

\section{Apps}
Als Apps (von engl. Applications), zu deutsch Anwendungen, werden Softwarepakete für jede Art von
Computer bezeichnet. Sie erfüllen meistens einen bestimmten Zweck und können von Anwendern auf dem
gewünschten Gerät installiert werden.

Computer-Anwendungen werden in fast allen Branchen der Wirtschaft eingesetzt, um Menschen bei der
Arbeit zu Unterstützen. Apps für Unterhaltung und Social Media sind wohl die Apps, welche am meisten
Bildschirmzeit von Anwendern beanspruchen.

Von einfachen ToDo-Listen-Apps bis zu komplexen Buchhaltungssystemen kann praktisch alles realisiert
werden, wofür die Leistung des Geräts reicht.

Die Umsetzung einer App-Idee kann durch mehrere Herangehensweisen angegangen werden.

\subsection{Native Apps}
Die Applikation wird direkt auf das Zielbetriebssystem aufgebaut und hat tiefen Zugriff auf die
Software- und Hardwareschnittstellen des Geräts. Es gibt jedoch meist viele Wege, also Sprachen und
Frameworks, um auf der Zielplattform die vorgegebene App zu erstellen.

\begin{itemize}
  \item Windows: C, C++, C\#
  \item MacOS und iOS: Swift, Objective-C
  \item Linux: C
  \item Android: Kotlin, Java, C++
\end{itemize}

\subsubsection{Vorteil}
Ein großer Vorteil davon ist die Performance. Es gibt keine Zwischen-Ebene, mit der kommuniziert
werden muss, sondern es wird die in das System eingebaute Render-Engine verwendet.

\subsubsection{Nachteil}
Die meisten Firmen wollen ihre Anwendungen auf mehreren Plattformen anbieten, bei einer
Smartphone-App also sowohl eine Android-App im Google Play Store, als auch eine iOS-App im App Store.
Um beides mit Native Apps zu realisieren benötigt man also zwei App-Entwicklungs-Teams, die zwei
verschiedene Repositories pflegen müssen und welche beide gleichzeitig neue Features einbauen und
testen müssen.

\subsection{Webanwendungen}
Eine mögliche Lösung ist die Webanwendung. Alles, was dafür auf dem Endgerät vorhanden sein muss,
ist ein Webbrowser, der die Webseite darstellen kann. Diese Webseite muss nun nur noch von einem
einzigen Team entwickelt und an unterschiedliche Bildschirmgrößen angepasst werden.

Durch die Verwendung einer Web-App im Browser geht jedoch ein großer Teil des Zugriffs auf die
gerätespezifische API verloren. Außerdem funktioniert eine Webanwendung nur mit einer aktiven
Internetverbindung und es muss bei jedem Screen-Wechsel ein neuer Web-Request verarbeitet werden.

\subsection{Hybrid-Apps}
In einer Hybrid-App werden Native App und Webanwendung kombiniert. Die Native App zeigt in einer
sogenannten WebView die gewünschte Webseite an und hat gleichzeitig auch Zugriff auf Software- und
Hardware-Schnittstellen. Der Endbenutzer sollte dabei nicht mitbekommen, dass die Anwendung
eigentlich nur einen Webbrowser darstellt. Beispiel: Ionic Framework.

\subsection{Cross-Platform-Apps}
Eine Cross-Platform-App hat genau wie eine Hybrid-App Zugriff auf tiefere Ebenen der Geräte-API,
der Unterschied besteht nur darin, dass die Benutzeroberfläche von einer einheitlichen
Auszeichnungssprache (Markup Language) wie XML bzw. XAML oder JSX in die jeweiligen UI-Elemente der
Zielplattform umgewandelt werden. Beispiele für Frameworks dieser Art sind: React Native, Xamarin.

\newpage
\section{Android: Entwicklung für die OpenSource-Welt}
\label{android}

\subsection{Was ist Android?}
\textit{Android} ist ein Betriebssystem, welches ursprünglich für die Steuerung von Digitalkameras
entwickelt wurde. Im Jahre 2008 veröffentlichte \textit{Google LLC} die erste Version des
Betriebssystems. Es sollte in den nachfolgenden Jahren von der \textit{Open Handset Alliance}, einem
Unternehmenszusammenschluss aus 84 Unternehmen (Stand 2017), weiterentwickelt werden und offene
Standards für die Mobilgeräte festlegen. \\ \\

Android 
\newpage
\section{iOS: Entwicklung für das Apple-Ökosystem}
\label{iosdev}

\subsection{Was ist iOS?}
iOS wird das Betriebssystem genannt, welches von Apple exklusiv für mobile Geräte entwickelt wird,
hauptsächlich für die iPhone-Produktreihe. Die aktuelle Version ist iOS 15.4, sie wurde am 20.
September 2021 veröffentlicht.

\subsection{Geschichte}
Die erste Version von iOS wurde im iPhone 1 im Jahr 2007 gezeigt. Das erste iPhone, welches am 9.
Januar 2007 von Steve Jobs, dem damaligen CEO von Apple, vorgestellt wurde, war die wohl
einflussreichste Erfindung des 21. Jahrhunderts. Es kombinierte die drei Funktionen Multimedia,
Telefonie und Internet in einem Gerät, welches noch dazu einfach zu bedienen war und keine physische
Tastatur benötigte, im Gegensatz zu vorherigen Smartphones wie das Motorola Q, BlackBerry oder Nokia
E62. Die Bedienung des Touch-Screens erfolgt auch nicht mit einem Eingabestift, sondern per Hand. So
gesehen war es das erste Mobiltelefon, welches wirklich den Namen Smartphone verdient hatte.

Das iPhone war sofort ein voller Erfolg und machte Apple zu einem der erfolgreichsten Unternehmen
der Welt. Nach der Veröffentlichung veränderte sich die Smartphone-Industrie drastisch, viele
Hersteller setzten im Design auf weniger Knöpfe und mehr Bildschirmfläche.

\subsection{App-Entwicklung}
Um Apps für das Apple-Ökosystem zu entwickeln, benötigt man natürlich einen PC, auf welchem MacOS
installiert ist. Die Programmierung erfolgt hauptsächlich in der Hauseigenen IDE von Apple, XCode,
und den Sprachen Objective-C und, seit 2014, auch mittels Swift.
\newpage
\section{React Native: Einmal lernen, überall schreiben}
\label{reactnative}

\subsection{Was ist React Native?}
React Native ist ein JavaScript-Framework, welches es ermöglicht mit Hilfe von JavaScript und der
dafür entwickelten Bibliothek \nameref{reactjs} Cross-Platform-Apps für Android, iOS, Windows und
Web zu entwickeln. Außerdem ist der Zugriff auf die native Plattform-API möglich.

\subsection{Core Components}
Die Erstellung der Benutzeroberfläche erfolgt hauptsächlich mit React Native Core Components. Diese
Core Components werden beim Kompilieren in die richtigen nativen Elemente umgewandelt.



\section{Expo CLI und Expo Go}
\subsection{Was ist Expo}


\section{React Native CLI}
\newpage
\section{Alternative Kandidaten}
\subsection{Flutter}
\label{flutter}
Flutter ist ein Framework, welches von Google erstellt wurde und heutzutage zu den modernsten seiner
Art gehört. Es benutzt für die Benutzeroberflächenerstellung die neuartige Sprache Dart, die extra
für dieses Framework entwickelt wurde und kann für die Erstellung von Cross-Platform-Applikationen
eingesetzt werden. Es unterstützt die Zielplattformen Android, iOS, Windows, MacOS, Linux, Web und
sogar Google Fuchsia \cite{flutter}.

\subsection{Xamarin}
\label{xamarin}
Das Cross-Plattform-Framework Xamarin ist, im Vergleich mit Xamarin und React Native, schon am
Längsten in Entwicklung. Es wird hauptsächlich in den Sprachen C\# und XAML verfasst und ist auch
für praktisch alle Zielplattformen geeignet. Jedoch gibt es einen entscheidenden Nachteil: Für jede
Zielplattform gibt es eine eigene Version des Frameworks. Es teilt sich auf in Xamarin\.Native und
Xamarin.Forms, letzteres unterstützt die Cross-Plattform Eigenschaften.

2020 wurde von Microsoft verkündet, dass sie eine Fusion aus Xamarin und .NET 6 planen, um ein
neues Framework zu erschaffen. Dieses soll den Namen .NET MAUI (.NET Multi-Platform App UI) tragen
\cite{xamarin}.
\newpage