\subsection{/acp/order/validate}
\label{validateOrder}

\begin{lstlisting}
    GET /acp/order/validate
\end{lstlisting}

\textbf{Beschreibung:} \\
Prüft die Validität einer Bestellung. Wird genutzt, um im Admin-Client die Bestellung aus einem gelesenen QR-Code zu überprüfen.

\textbf{Authorization-Typ:} \\
ACP

\subsubsection{Parameter}

\lstinline{order: string} (benötigt) \\
$\rightarrow$ Die zu validierende Bestellung im Format \glqq userId:orderId\grqq

\subsubsection{Beispiel}

\begin{lstlisting}
    GET /acp/order/get?validate?order=25:10
\end{lstlisting}

Validiert die Bestellung mit der ID \lstinline{10} vom Nutzer \lstinline{25}.

\subsubsection{Antwort (bei Erfolg)}

\lstinline{Content-Type: application/json}
\begin{lstlisting}
    {
        "success": true, 
        "valid": true,
        "order": {
            "id": 60,
            "user_id": 26,
            "timestamp": "2021-03-01T08:29:34.000Z"
            "state": "VALID",
            "rebate": 10,
            "total": 6,
            "menuOrders": [
                "menu_id": 1,
                "quantity": 1,
                "menu": <Menu Object>
            ],
            "productOrders": [
                "product_id": 1,
                "quantity": 1,
                "product": <Product Object>
            ]
        }
    }
\end{lstlisting}

\subsubsection{Antwort (bei ungültiger Bestellung)}

\lstinline{Content-Type: application/json}
\begin{lstlisting}
    {
        "success": true, 
        "valid": false,
        "reasons": [
            "Order does not belong to user",
            "Order was not created yesterday",
            "Order has already been invalidated"
        ]
    }
\end{lstlisting}

Wobei \lstinline{reasons} immer ein String-Array mit allen zutreffenden Begründungen ist, weshalb der übertragene Code ungültig ist.