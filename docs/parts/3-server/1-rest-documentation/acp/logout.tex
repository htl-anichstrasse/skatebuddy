\subsection{/acp/logout}

\begin{lstlisting}
    POST /acp/logout
\end{lstlisting}

\textbf{Beschreibung:} \\
Invalidiert einen Session-Token eines ACP-Nutzerkontos und führt damit eine Abmeldung eines ACP-Nutzerkontos im ACP durch.

\textbf{Authorization-Typ:} \\
Keine

\subsubsection{Payload}

\lstinline{name: string} (benötigt) \\
$\rightarrow$ Der Name des ACP-Nutzerkontos, bei dem die Abmeldung durchgeführt werden soll
\lstinline{token: string} (benötigt) \\
$\rightarrow$ Der Session-Token des ACP-Nutzerkontos, der invalidiert werden soll.

\subsubsection{Beispiel}

\begin{lstlisting}
    POST /acp/logout
    {
        "name": "admin",
        "token": "7sGCYBi5796mvPYe1Jg5pybMuJXBGa%2B4"
    }
\end{lstlisting}

Führt eine Anmeldung für das ACP-Nutzerkonto \lstinline{admin} mit dem Passwort \lstinline{securepassword123} durch und returniert einen generierten Session-Token.

\subsubsection{Antwort (bei Erfolg)}

\lstinline{Content-Type: application/json}
\begin{lstlisting}
    {
        "success": true, 
        "message": "Successfully invalidated session"
    }
\end{lstlisting}
