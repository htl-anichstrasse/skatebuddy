\subsection{/user/validate}

\begin{lstlisting}
    POST /user/validate
\end{lstlisting}

\textbf{Beschreibung:} \\
Prüft, ob der Session-Token eines Nutzerkontos gültig ist. Wird genutzt, um bei Aufruf des Clients einen vorhandenen Session-Cookie zu prüfen.

\textbf{Authorization-Typ:} \\
App

\subsubsection{Payload}

\lstinline{email: string} (benötigt) \\
$\rightarrow$ Die E-Mail-Adresse des Nutzerkontos, bei dem die Validierung durchgeführt werden soll

\lstinline{token: string} (benötigt) \\
$\rightarrow$ Der Token des Nutzerkontos, bei dem die Validierung durchgeführt werden soll

\subsubsection{Beispiel}

\begin{lstlisting}
    POST /user/validate
    {
        "email": "meineEmail@sokka.me",
        "token": "eYiKOYqtihKgVknurEq1luc3mhawIxAo"
    }
\end{lstlisting}

Validiert den Session-Token \lstinline{eYiKOYqtihKgVknurEq1luc3mhawIxAo} für das Nutzerkonto mit der E-Mail-Adresse \lstinline{meineEmail@sokka.me}.

\subsubsection{Antwort (bei Erfolg)}

\lstinline{Content-Type: application/json}
\begin{lstlisting}
    {
        "success": true, 
        "message": "Token for this email is valid"
    }
\end{lstlisting}

\subsubsection{Antwort (bei ungültigen Token)}

\lstinline{Content-Type: application/json}
\begin{lstlisting}
    {
        "success": true, 
        "message": "Could not validate token for email 'meineEmail@sokka.me'"
    }
\end{lstlisting}
