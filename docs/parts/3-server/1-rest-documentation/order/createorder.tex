\subsection{/order/create}

\begin{lstlisting}
    POST /order/create
\end{lstlisting}

\textbf{Beschreibung:} \\
Erstellt eine neue Bestellung. Wird genutzt, um neue Bestellungen im Client zu tätigen.

\textbf{Authorization-Typ:} \\
App

\subsubsection{Payload}

\lstinline{products: ProductOrder[]} (benötigt) \\
$\rightarrow$ Der Produktbestellungen dieser Bestellung

\lstinline{menus: MenuOrder[]} (benötigt) \\
$\rightarrow$ Die Menübestellungen dieser Bestellung

\subsubsection{Beispiel}

\begin{lstlisting}
    POST /order/create
    {
        "products": [
            {
                "product_id": 3,
                "quantity": 2
            }
        ],
        "menus": [
            {
                "menu_id": 2,
                "quantity": 1
            }
        ]
    }
\end{lstlisting}

Tätigt eine Bestellung von 2x Produkt mit ID \lstinline{3} und 1x Menü mit ID \lstinline{2}.

\subsubsection{Antwort (bei Erfolg)}

\lstinline{Content-Type: application/json}
\begin{lstlisting}
    {
        "success": true, 
        "message": "Order successfully created"
    }
\end{lstlisting}