\subsection{authContext}
\begin{code}[htp]
\begin{lstlisting}[firstnumber=1,language=JavaScript, style=JSX]
const authContext = useMemo(
  () => ({
    restoreToken: async token => {
      const res = await fetch(
        'https://skate-buddy.josholaus.com/api/users/validate',
        {
          method: 'POST',
          headers: {
            Accept: 'application/json',
            'Content-Type': 'application/json',
            Authorization: `Bearer ${token}`,
          },
          body: JSON.stringify({ token }),
        },
      );

      if (res.status === 200) {
        authContext.decodeToken(token);
        dispatch({ type: 'RESTORE_TOKEN', token });
      } else {
        dispatch({ type: 'SIGN_OUT' });
      }
    },
    signIn: async data => {
      ...
    },
    ...
  }),
  [],
);
\end{lstlisting}
\caption{JavaScript Funktion - Die Auth-Funktionen}
\end{code}

In authContext werden alle Funktionen erstellt, welche für die Server-Anfragen der Authentifikation
wichtig sind. RestoreToken ist die Funktion, die als erstes aufgerufen wird, nachdem die App
gestartet wurde. Sie schickt eine POST-Anfrage an den Server (Zeile 4) und übergibt im Content den
Token, den es zu überprüfen gilt (Zeile 13). Diese Anfrage wird mit dem await-Statement abgewartet.
Nachdem die Antwort eingetroffen ist, wird noch die Information aus dem Token entnommen (Zeile 18)
und anschließend abgespeichert, mit dem dispatch-Aufruf in Zeile 19. Sollte der Token nicht gültig
oder die Anfrage einfach nicht funktioniert haben, so wir der Benutzer ausgeloggt (Zeile 21).
