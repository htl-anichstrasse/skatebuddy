\chapter{Authentifizierung}
Die Umsetzung der Authentifizierung war das wohl komplizierteste Thema in der gesamten
Entwicklungsphase. Um die Authentifizierung in der App reibungslos durchzuführen, werden mehrere
React Hooks verwendet.

Im Einstiegspunkt der App, also App.js, ist werden mehrere Bibliotheken eingebunden, die in der
ganzen App verwendet werden.

\begin{code}[htp]
\begin{lstlisting}[firstnumber=1,language=JavaScript, style=JSX]
const App = () => (
  <AuthProvider>
    <NavigationContainer>
      <AuthHandler />
    </NavigationContainer>
    <FlashMessage position="top" />
  </AuthProvider>
);

export default App;
\end{lstlisting}
\caption{React Component - AuthProvider ist die äußerste Komponente.}
\end{code}

\newpage
\section{AuthContext}
AuthContext.js ist für die Speicherung, Änderung und Überprüfung des Json Web Token zuständig. Die
Datei besteht aus drei Abschnitten:

\subsection{state \& dispatch}
useReducer ist ein React-Hook, welcher in Kombination mit dem Hook useContext eine Alternative zur
React-Bibliothek Redux darstellt. Redux wurde entwickelt, um die Speicherung und komplizierte
Änderung von Zuständen in React Anwendungen zu vereinfachen. Redux wurde jedoch sehr komplex und
benötigte viel Code, um zu funktionieren, daher entwickelten die Ersteller von Redux eine
zusätzliche Wrapper-Bibliothek, Redux-Toolkit, um das Erstellen von Redux-Stores noch einmal zu
vereinfachen \cite{reduxToolkit}. Stores werden in Apps Variablen oder Objekte genannt, die von
überall in der App aufrufbar sein sollen und zentrale Informationen zum Zustand speichern.

\begin{code}[htp]
\begin{lstlisting}[firstnumber=1,language=JavaScript, style=JSX]
const [state, dispatch] = useReducer(
  (prevState, action) => {
    switch (action.type) {
      case 'RESTORE_TOKEN':
        return {
          userToken: action.token,
          isLoading: false,
        };
      case 'SIGN_OUT':
        return {
          currentUser: null,
          userToken: null,
          isLoading: false,
        };
      ...
    }
  },
  {
    isLoading: true,
    userToken: null,
    currentUser: null,
  },
);
\end{lstlisting}
\caption{JavaScript Funktion - Aus dem Hook werden zwei Variablen extrahiert, state und dispatch.}
\end{code}

State ist das Objekt, welches wir als zweiten Parameter dem Hook übergeben. Dispatch ist eine
Funktion, welche aufgerufen werden kann, um den State zu verändern.

\newpage

\begin{code}[htp]
\begin{lstlisting}[firstnumber=1,language=JavaScript, style=JSX]
dispatch({ type: 'SIGN_OUT' });
\end{lstlisting}
\caption{JavaScript Funktion - Die State-Veränderung "SIGN-OUT" wird aufgerufen.}
\end{code}

Dispatch wird ein Objekt übergeben, welches im useReducer-Hook action genannt wird. Als erstes wird
anhand von action.type in Zeile 3 überprüft, welche Art von Veränderung vorgenommen werden soll.
Der Wert der aus der Funktion zurückgegeben wird, wird als neuer State abgespeichert. Es ist auch
möglich auf den vorherigen Zustand zuzugreifen, über die Variable prevState.

\newpage

\subsection{authContext}
\begin{code}[htp]
\begin{lstlisting}[firstnumber=1,language=JavaScript, style=JSX]
const authContext = useMemo(
  () => ({
    restoreToken: async token => {
      const res = await fetch(
        'https://skate-buddy.josholaus.com/api/users/validate',
        {
          method: 'POST',
          headers: {
            Accept: 'application/json',
            'Content-Type': 'application/json',
            Authorization: `Bearer ${token}`,
          },
          body: JSON.stringify({ token }),
        },
      );

      if (res.status === 200) {
        authContext.decodeToken(token);
        dispatch({ type: 'RESTORE_TOKEN', token });
      } else {
        dispatch({ type: 'SIGN_OUT' });
      }
    },
    signIn: async data => {
      ...
    },
    ...
  }),
  [],
);
\end{lstlisting}
\caption{JavaScript Funktion - Die Auth-Funktionen}
\end{code}

In authContext werden alle Funktionen erstellt, welche für die Server-Anfragen der Authentifikation
wichtig sind. RestoreToken ist die Funktion, die als erstes aufgerufen wird, nachdem die App
gestartet wurde. Sie schickt eine POST-Anfrage an den Server (Zeile 4) und übergibt im Content den
Token, den es zu überprüfen gilt (Zeile 13). Diese Anfrage wird mit dem await-Statement abgewartet.
Nachdem die Antwort eingetroffen ist, wird noch die Information aus dem Token entnommen (Zeile 18)
und anschließend abgespeichert, mit dem dispatch-Aufruf in Zeile 19. Sollte der Token nicht gültig
oder die Anfrage einfach nicht funktioniert haben, so wir der Benutzer ausgeloggt (Zeile 21).

\newpage

\subsection{AuthProvider}
Alle diese Funktionen und Variablen sind enthalten in der Komponente AuthProvider. Sie übernimmt die
wichtigste Aufgabe von allen, nämlich das Bereitstellen all dieser Funktionalität an die restliche
App.

\begin{lstlisting}
const AuthContext = React.createContext();
const AuthContextState = React.createContext();

const useAuthContext = () => useContext(AuthContext);
const useAuthContextState = () => useContext(AuthContextState);

const AuthProvider = ({ children }) => {
  const [state, dispatch] = useReducer(
    ...
  );

  const authContext = useMemo(
    ...
  );

  return (
    <AuthContext.Provider value={authContext}>
      <AuthContextState.Provider value={state}>
        {children}
      </AuthContextState.Provider>
    </AuthContext.Provider>
  );
};

export { AuthProvider };
export { useAuthContext };
export { useAuthContextState };
\end{lstlisting}

AuthProvider umschließt in App.js alle anderen Komponenten, hier wird offensichtlich warum das so
sein muss. Zuerst muss aber geklärt werden, was ein Context ist.

In Zeile 3 wird als erstes ein Context erstellt. Dieser erhält den Namen AuthContext und soll die
Funktionen aus authContext in der restlichen App verfügbar machen.

In Zeile 19 wird über AuthContext.Provider ein ContextProvider erstellt, der eine Variable über das
Property value annimmt.

Wird nun in einer Komponente, die ein Kind von AuthContext ist, die Funktion useContext(AuthContext)
aufgerufen, so ist der zurückgegebene Wert gleich der value, die dem Provider übergeben wurde.

Um zu umgehen, dass ich jedes mal, wenn ich eine Funktion brauche, useContext und zusätzlich auch
noch AuthContext importieren muss, fasse ich diesen Ausdruck zu useAuthContext zusammen (Zeile 6).
So ist es möglich, durch das importieren von useAuthContext in einem Kind von AuthContext.Provider
auf die gespeicherte Variable zuzugreifen.

In einer üblichen Anwendung werden Daten über Props von Eltern an Kinder weitergegeben. Context
macht es möglich direkt für alle Kinder eine Variable zur Verfügung zu stellen, ohne jedem einzeln
die Variable zu übergeben.




\newpage
\section{AuthHandler}
In AuthHandler entscheidet die App, ob ein Benutzer eingeloggt ist oder nicht.

\begin{lstlisting}
const Auth = () => {
  const { restoreToken } = useAuthContext();
  const state = useAuthContextState();

  useEffect(() => {
    const checkForValidToken = async () => {
      let token = null;
      if (await Keychain.hasInternetCredentials('jwt')) {
        const creds = await Keychain.getInternetCredentials('jwt');
        token = creds.password;
      }
      restoreToken(token);
    };

    checkForValidToken();
  }, [restoreToken]);

  if (state.isLoading) return <SplashScreen />;
  if (state.userToken == null) return <LoginSignupStack />;

  return <BottomTabsNavigator />;
};

export default Auth;
\end{lstlisting}

Als erstes werden die Funktion restoreToken und state aus dem Content entnommen. Sobald die App
gestartet wird, wird die asynchrone Funktion checkForValidToken aufgerufen. In ihr wird in Zeile 10
und 11 zuerst überprüft, ob im verschlüsselten, internen Speicher ein Eintrag für "jwt" besteht.

Die Komponente Keychain wird von der Bibliothek react-native-keychain bereitgestellt. Sie enthält
Funktionen um auf Android und iOS sensible Informationen sicher abzuspeichern. Auf Android wird der
Android Keystore verwendet, auf iOS die Keychain Services.

Die Keychain ist in unserer App dafür zuständig, den JWT auch nach dem Beenden der App im internen
Speicher festzuhalten und beim Start wieder daraus auszulesen. Sollte ein Eintrag vorhanden sein,
wird der Token der Funktion restoreToken aus authContext übergeben, in welcher er überprüft wird.

Beim Start der App wurde state.isLoading als true und state.userToken als null definiert. Solange
isLoading auf true ist, wird der Ladebildschirm SplashScreen angezeigt. Sobald der Token überprüft
wurde, wird isLoading auf false gesetzt und es entscheidet sich ob der User eingeloggt wird, oder
nicht.

Ist nach der Überprüfung die Variable userToken immer noch null, so wird der LoginStack angezeigt.
Sollte der Token gültig sein, so wird er in userToken abgespeichert, was die If-Abfrage in Zeile 27
false macht. Somit bleibt nur noch das Statement in Zeile 31 übrig, der User ist eingeloggt.



