\section{AuthHandler}
In \textbf{AuthHandler} entscheidet die App, ob ein Benutzer eingeloggt ist oder nicht.

\begin{code}[htp]
\begin{lstlisting}[firstnumber=1,language=JavaScript, style=JSX]
const Auth = () => {
  const { restoreToken } = useAuthContext();
  const state = useAuthContextState();

  useEffect(() => {
    const checkForValidToken = async () => {
      let token = null;
      if (await Keychain.hasInternetCredentials('jwt')) {
        const creds = await Keychain.getInternetCredentials('jwt');
        token = creds.password;
      }
      restoreToken(token);
    };

    checkForValidToken();
  }, [restoreToken]);

  if (state.isLoading) return <SplashScreen />;
  if (state.userToken == null) return <LoginSignupStack />;

  return <BottomTabsNavigator />;
};

export default Auth;
\end{lstlisting}
\caption{React Component - Ob ein Benutzer eingeloggt ist, hängt von state.userToken ab.}
\end{code}

Als erstes werden die Funktion \textbf{restoreToken} und \textbf{state} aus dem Context entnommen.
Sobald die App gestartet wird, wird die asynchrone Funktion \textbf{checkForValidToken} aufgerufen.
In ihr wird in Zeile 10 und 11 zuerst überprüft, ob im verschlüsselten, internen Speicher ein
Eintrag für \textbf{jwt} besteht.

Die Komponente \textbf{Keychain} wird von der Bibliothek \textbf{react-native-keychain}
bereitgestellt. Sie enthält Funktionen um auf Android und iOS sensible Informationen sicher
abzuspeichern. Auf Android wird der \textbf{Android Keystore} verwendet, auf iOS die
\textbf{Keychain Services}.

Die \textbf{Keychain} ist in unserer App dafür zuständig, den JWT auch nach dem Beenden der App im
internen Speicher festzuhalten und beim Start wieder daraus auszulesen. Sollte ein Eintrag vorhanden
sein, wird der Token der Funktion \textbf{restoreToken} aus \textbf{authContext} übergeben, in
welcher er überprüft wird.

\newpage
Beim Start der App wurde \textbf{state.isLoading} als \textbf{true} und \textbf{state.userToken} als
\textbf{null} definiert. Solange \textbf{isLoading} auf \textbf{true} ist, wird der Ladebildschirm
(\textbf{SplashScreen.js}) angezeigt. Sobald der Token überprüft wurde, wird \textbf{isLoading} auf
\textbf{false} gesetzt und es entscheidet sich ob der User eingeloggt wird, oder nicht.

Ist nach der Überprüfung die Variable \textbf{userToken} immer noch \textbf{null}, so wird der
\textbf{LoginStack} angezeigt. Sollte der Token gültig sein, so wird er in v{userToken}
abgespeichert, was die If-Abfrage in Zeile 27 \textbf{false} macht. Somit bleibt nur noch das
Statement in Zeile 31 übrig, der User ist eingeloggt.
