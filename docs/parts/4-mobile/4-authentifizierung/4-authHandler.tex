\section{AuthHandler}
In AuthHandler entscheidet die App, ob ein Benutzer eingeloggt ist oder nicht.

\begin{code}[htp]
\begin{lstlisting}[firstnumber=1,language=JavaScript, style=JSX]
const Auth = () => {
  const { restoreToken } = useAuthContext();
  const state = useAuthContextState();

  useEffect(() => {
    const checkForValidToken = async () => {
      let token = null;
      if (await Keychain.hasInternetCredentials('jwt')) {
        const creds = await Keychain.getInternetCredentials('jwt');
        token = creds.password;
      }
      restoreToken(token);
    };

    checkForValidToken();
  }, [restoreToken]);

  if (state.isLoading) return <SplashScreen />;
  if (state.userToken == null) return <LoginSignupStack />;

  return <BottomTabsNavigator />;
};

export default Auth;
\end{lstlisting}
\caption{React Component - Ob ein Benutzer eingeloggt ist, hängt von state.userToken ab.}
\end{code}

Als erstes werden die Funktion restoreToken und state aus dem Content entnommen. Sobald die App
gestartet wird, wird die asynchrone Funktion checkForValidToken aufgerufen. In ihr wird in Zeile 10
und 11 zuerst überprüft, ob im verschlüsselten, internen Speicher ein Eintrag für "jwt" besteht.

Die Komponente Keychain wird von der Bibliothek react-native-keychain bereitgestellt. Sie enthält
Funktionen um auf Android und iOS sensible Informationen sicher abzuspeichern. Auf Android wird der
Android Keystore verwendet, auf iOS die Keychain Services.

Die Keychain ist in unserer App dafür zuständig, den JWT auch nach dem Beenden der App im internen
Speicher festzuhalten und beim Start wieder daraus auszulesen. Sollte ein Eintrag vorhanden sein,
wird der Token der Funktion restoreToken aus authContext übergeben, in welcher er überprüft wird.

Beim Start der App wurde state.isLoading als true und state.userToken als null definiert. Solange
isLoading auf true ist, wird der Ladebildschirm SplashScreen angezeigt. Sobald der Token überprüft
wurde, wird isLoading auf false gesetzt und es entscheidet sich ob der User eingeloggt wird, oder
nicht.

Ist nach der Überprüfung die Variable userToken immer noch null, so wird der LoginStack angezeigt.
Sollte der Token gültig sein, so wird er in userToken abgespeichert, was die If-Abfrage in Zeile 27
false macht. Somit bleibt nur noch das Statement in Zeile 31 übrig, der User ist eingeloggt.
