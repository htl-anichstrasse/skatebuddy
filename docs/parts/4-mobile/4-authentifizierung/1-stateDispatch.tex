\subsection{state \& dispatch}
useReducer ist ein React-Hook, welcher in Kombination mit dem Hook useContext eine Alternative zur
React-Bibliothek Redux darstellt. Redux wurde entwickelt, um die Speicherung und komplizierte
Änderung von Zuständen in React Anwendungen zu vereinfachen. Redux wurde jedoch sehr komplex und
benötigte viel Code, um zu funktionieren, daher entwickelten die Ersteller von Redux eine
zusätzliche Wrapper-Bibliothek, Redux-Toolkit, um das Erstellen von Redux-Stores noch einmal zu
vereinfachen \cite{reduxToolkit}. Stores werden in Apps Variablen oder Objekte genannt, die von
überall in der App aufrufbar sein sollen und zentrale Informationen zum Zustand speichern.

\begin{code}[htp]
\begin{lstlisting}[firstnumber=1,language=JavaScript, style=JSX]
const [state, dispatch] = useReducer(
  (prevState, action) => {
    switch (action.type) {
      case 'RESTORE_TOKEN':
        return {
          userToken: action.token,
          isLoading: false,
        };
      case 'SIGN_OUT':
        return {
          currentUser: null,
          userToken: null,
          isLoading: false,
        };
      ...
    }
  },
  {
    isLoading: true,
    userToken: null,
    currentUser: null,
  },
);
\end{lstlisting}
\caption{JavaScript Funktion - Aus dem Hook werden zwei Variablen extrahiert, state und dispatch.}
\end{code}

State ist das Objekt, welches wir als zweiten Parameter dem Hook übergeben. Dispatch ist eine
Funktion, welche aufgerufen werden kann, um den State zu verändern.

\newpage

\begin{code}[htp]
\begin{lstlisting}[firstnumber=1,language=JavaScript, style=JSX]
dispatch({ type: 'SIGN_OUT' });
\end{lstlisting}
\caption{JavaScript Funktion - Die State-Veränderung "SIGN-OUT" wird aufgerufen.}
\end{code}

Dispatch wird ein Objekt übergeben, welches im useReducer-Hook action genannt wird. Als erstes wird
anhand von action.type in Zeile 3 überprüft, welche Art von Veränderung vorgenommen werden soll.
Der Wert der aus der Funktion zurückgegeben wird, wird als neuer State abgespeichert. Es ist auch
möglich auf den vorherigen Zustand zuzugreifen, über die Variable prevState.
