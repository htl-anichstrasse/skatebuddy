\subsection{AuthProvider}
Alle diese Funktionen und Variablen sind enthalten in der Komponente AuthProvider. Sie übernimmt die
wichtigste Aufgabe von allen, nämlich das Bereitstellen all dieser Funktionalität an die restliche
App.

\begin{code}[htp]
\begin{lstlisting}[firstnumber=1,language=JavaScript, style=JSX]
const AuthContext = React.createContext();
const AuthContextState = React.createContext();

const useAuthContext = () => useContext(AuthContext);
const useAuthContextState = () => useContext(AuthContextState);

const AuthProvider = ({ children }) => {
  const [state, dispatch] = useReducer(
    ...
  );

  const authContext = useMemo(
    ...
  );

  return (
    <AuthContext.Provider value={authContext}>
      <AuthContextState.Provider value={state}>
        {children}
      </AuthContextState.Provider>
    </AuthContext.Provider>
  );
};

export { AuthProvider };
export { useAuthContext };
export { useAuthContextState };
\end{lstlisting}
\caption{React Component - Alle Kinder erhalten Zugriff auf authContext und state}
\end{code}

\textbf{AuthProvider} umschließt in \textbf{App.js} alle anderen Komponenten, hier wird
offensichtlich, warum das so sein muss. Zuerst muss aber geklärt werden, was ein Context ist.

In Zeile 3 wird als erstes ein Context erstellt. Dieser erhält den Namen AuthContext und soll die
Funktionen aus authContext in der restlichen App verfügbar machen.

In Zeile 19 wird über \textbf{AuthContext.Provider} ein ContextProvider erstellt, der eine Variable über das
Property \textbf{value} annimmt.

Wird nun in einer Komponente, die ein Kind von \textbf{AuthContext} ist, die Funktion
\textbf{useContext(AuthContext)} aufgerufen, so ist der zurückgegebene Wert gleich der \textbf{value}, die
dem Provider übergeben wurde.

\newpage
Um zu umgehen, dass ich jedes mal, wenn ich eine Funktion brauche, useContext und zusätzlich auch
noch \textbf{AuthContext} importieren muss, fasse ich diesen Ausdruck zu \textbf{useAuthContext}
zusammen (Zeile 6). So ist es möglich, durch das Importieren von \textbf{useAuthContext} in einem
Kind von \textbf{AuthContext.Provider} auf die gespeicherte Variable zuzugreifen.

In einer üblichen Anwendung werden Daten über Props von Eltern an Kinder weitergegeben. Context
macht es möglich direkt für alle Kinder eine Variable zur Verfügung zu stellen, ohne jedem einzeln
die Variable übergeben zu müssen.