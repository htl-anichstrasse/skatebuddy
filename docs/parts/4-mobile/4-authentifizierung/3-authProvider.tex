\subsection{AuthProvider}
Alle diese Funktionen und Variablen sind enthalten in der Komponente AuthProvider. Sie übernimmt die
wichtigste Aufgabe von allen, nämlich das Bereitstellen all dieser Funktionalität an die restliche
App.

\begin{lstlisting}
const AuthContext = React.createContext();
const AuthContextState = React.createContext();

const useAuthContext = () => useContext(AuthContext);
const useAuthContextState = () => useContext(AuthContextState);

const AuthProvider = ({ children }) => {
  const [state, dispatch] = useReducer(
    ...
  );

  const authContext = useMemo(
    ...
  );

  return (
    <AuthContext.Provider value={authContext}>
      <AuthContextState.Provider value={state}>
        {children}
      </AuthContextState.Provider>
    </AuthContext.Provider>
  );
};

export { AuthProvider };
export { useAuthContext };
export { useAuthContextState };
\end{lstlisting}

AuthProvider umschließt in App.js alle anderen Komponenten, hier wird offensichtlich warum das so
sein muss. Zuerst muss aber geklärt werden, was ein Context ist.

In Zeile 3 wird als erstes ein Context erstellt. Dieser erhält den Namen AuthContext und soll die
Funktionen aus authContext in der restlichen App verfügbar machen.

In Zeile 19 wird über AuthContext.Provider ein ContextProvider erstellt, der eine Variable über das
Property value annimmt.

Wird nun in einer Komponente, die ein Kind von AuthContext ist, die Funktion useContext(AuthContext)
aufgerufen, so ist der zurückgegebene Wert gleich der value, die dem Provider übergeben wurde.

Um zu umgehen, dass ich jedes mal, wenn ich eine Funktion brauche, useContext und zusätzlich auch
noch AuthContext importieren muss, fasse ich diesen Ausdruck zu useAuthContext zusammen (Zeile 6).
So ist es möglich, durch das importieren von useAuthContext in einem Kind von AuthContext.Provider
auf die gespeicherte Variable zuzugreifen.

In einer üblichen Anwendung werden Daten über Props von Eltern an Kinder weitergegeben. Context
macht es möglich direkt für alle Kinder eine Variable zur Verfügung zu stellen, ohne jedem einzeln
die Variable zu übergeben.