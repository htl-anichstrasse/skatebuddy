\section{Screens}
In unserer App wollen wir Informationen für den Benutzer verständlich darstellen. Es würde keinen
Sinn machen alle Informationen auf die gleiche Seite zu schreiben, eine logische Gliederung der App
ist daher sehr wichtig. Screens sind im Grunde nur React-Komponenten; sie werden aus mehreren
kleineren Komponenten zusammengebaut.

\section{Arten von Screens}
In Apps gibt es eigentlich nicht so viele Arten von Screens, hier die häufigsten Anwendungsfälle im
Überblick:

\subsection{Listen}
Listen-Screens sind dafür da, um mehrere Datensätze untereinander oder nebeneinander darzustellen.
Dafür wird über jedes Element in einer Liste iteriert und an eine Komponente weitergegeben, welche
anschließend die Daten aufbereitet und darstellt.

\subsection{Details}
Nachdem man jedes Element aufgelistet hat, sollte es noch möglich sein ein Element genauer zu
betrachten bzw. alle Informationen anzuzeigen, welche nicht im Überblick der Liste Platz haben.

\subsection{Formulare}
In Formularen kann der Benutzer Informationen in die App eintragen, welche dann per \textbf{HTTP} an
das Backend geschickt werden.