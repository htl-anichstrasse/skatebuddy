\subsection{SkateparksStack}
Der SkateparksStack ist zusammengesetzt aus SkateparksList

\section{SkateparksList}
Auf der ersten Seite sollte man sofort einen Überblick über alle Skateparks erhalten, die wir in
unserer Datenbank gespeichert haben.

\begin{figure}[H]
  \begin{center}
    \includegraphics[width=0.6\textwidth]{Mobile/Skateparks/SkateparksList.png}
    \caption{SkateparksList Screen}
  \end{center}
\end{figure}

Wenn der Benutzer auf einen Skatepark drückt, wird bei der Navigation zu SkateparkDetails der
Skatepark übergeben, um die Informationen anzuzeigen.

Am oberen Rand des Bildschirms hat der Benutzer die Möglichkeit, nach einem bestimmten
Skateparknamen zu suchen und nach der Reisezeit zu sortieren, aufsteigend und absteigend.

\begin{figure}[H]
  \begin{center}
    \includegraphics[width=0.6\textwidth]{Mobile/Skateparks/Search.png}
    \caption{Suchfunktion Demonstration}
  \end{center}
\end{figure}

\begin{figure}[H]
  \begin{center}
    \includegraphics[width=0.6\textwidth]{Mobile/Skateparks/SortModal.png}
    \caption{Sortieren nach Zeit}
  \end{center}
\end{figure}

\section{SkateparkDetails}
Als erstes werden die wichtigsten Informationen zum Standort des Parks angezeigt. Darunter gibt es
eine kleine Diashow mit diversen Bildern des Parks.





