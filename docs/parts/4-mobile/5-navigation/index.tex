\chapter{Navigation}
Für die Navigation innerhalb der App wurde auf intuitives und simples Design gesetzt, es sollte von
jedem sofort verstanden werden. Natürlich ist das Selbst-Implementieren dieser Funktionalität keine
Option, daher wurde die Bibliothek React Navigation verwendet. Sie verknüpft die einzelnen Screens
der App miteinander und stellt das Gegenstück zu React Router für React im Webbrowser dar.

Damit die Navigation funktionieren kann, muss die gesamte App ein Kind von NavigationContainer sein,
einer Komponente, die von React Navigation importiert wird.

\begin{lstlisting}
import { NavigationContainer } from '@react-navigation/native';

const App = () => (
  <AuthProvider>
    <NavigationContainer>
      <AuthHandler />
    </NavigationContainer>
    <FlashMessage position="top" />
  </AuthProvider>
);
\end{lstlisting}

Die Navigation innerhalb der App wurde auf zwei Arten umgesetzt:

\section{Tab-Navigation}
Andere mit React Native gebauten Apps, wie Facebook oder Instagram, verwenden eine sogenannte
"Bottom-Tabs"-Navigation, also eine Tab-Basierte Navigation auf der Unterseite des Bildschirms, um
ihre Hauptfunktionen dem Benutzer zu präsentieren.

Um die App simpel zu halten ist dies auch unsere präferierte Navigations-Methode, um die einzelnen
Bereiche der Anwendung miteinander zu verbinden.

\begin{lstlisting}
import MapScreen from '../screens/MapScreen';
import ProfileScreen from '../screens/ProfileScreen';
import SkateparksStack from './SkateparksStack';
import Colors from '../styles/Colors';

const Tab = createBottomTabNavigator();

const BottomTabsNavigator = () => {
  const tabBarIcons = (route, focused, color) => {
    ...
  };

  return (
    <Tab.Navigator
      backBehavior="initialRoute"
      initialRouteName="Skateparks"
      screenOptions={({ route }) => ({
        tabBarIcon: ({ focused, color }) => tabBarIcons(route, focused, color),
        tabBarActiveTintColor: Colors.primary,
        tabBarInactiveTintColor: Colors.gray2,
        tabBarActiveBackgroundColor: Colors.primarySoft,
        tabBarShowLabel: false,
        headerShown: false,
      })}
    >
      <Tab.Screen name="Skateparks" component={SkateparksStack} />
      <Tab.Screen name="Map" component={MapScreen} />
      <Tab.Screen name="Profile" component={ProfileScreen} />
    </Tab.Navigator>
  );
};

export default BottomTabsNavigator;
\end{lstlisting}



\newpage

\section{Stack-Navigation}


\begin{lstlisting}
import SkateparksList from '../screens/SkateparksList';
import SkateparkDetails from '../screens/SkateparkDetails';

const Stack = createNativeStackNavigator();

const SkateparksStack = () => (
  <Stack.Navigator
    initialRouteName="SkateparksList"
    screenOptions={{
      headerShown: false,
    }}
  >
    <Stack.Screen name="SkateparksList" component={SkateparksList} />
    <Stack.Screen name="SkateparkDetails" component={SkateparkDetails} />
  </Stack.Navigator>
);

export default SkateparksStack;
\end{lstlisting}