\chapter{Navigation}
Für die Navigation innerhalb der App wurde auf intuitives und simples Design gesetzt, es sollte von
jedem sofort verstanden werden. Natürlich ist das Selbst-Implementieren dieser Funktionalität keine
Option, daher wurde die Bibliothek React Navigation verwendet. Sie verknüpft die einzelnen Screens
der App miteinander und stellt das Gegenstück zu React Router für React im Webbrowser dar.

Damit die Navigation funktionieren kann, muss die gesamte App ein Kind von NavigationContainer sein,
einer Komponente, die von React Navigation importiert wird.

\begin{code}[htp]
\begin{lstlisting}[firstnumber=1,language=JavaScript, style=JSX]
import { NavigationContainer } from '@react-navigation/native';

const App = () => (
  <AuthProvider>
    <NavigationContainer>
      <AuthHandler />
    </NavigationContainer>
    <FlashMessage position="top" />
  </AuthProvider>
);
\end{lstlisting}
\caption{JavaScript Funktion - In NavigationContainer können Navigationen erstellt werden.}
\end{code}

Die Navigation innerhalb der App wurde auf zwei Arten umgesetzt:

\begin{figure}[H]
  \begin{center}
    \includegraphics[width=0.5\textwidth]{Mobile/Navigators.png}
    \caption{Stack Navigator und Tab-Navigator}
  \end{center}
\end{figure}

\newpage
\input{parts/0-preamble/index}
\input{parts/1-intro/index}
\input{parts/2-theorie/index}
\input{parts/3-server/index}
\input{parts/5-mobile/index}
\input{parts/7-appendix/index}
\input{parts/0-preamble/index}
\input{parts/1-intro/index}
\input{parts/2-theorie/index}
\input{parts/3-server/index}
\input{parts/5-mobile/index}
\input{parts/7-appendix/index}
\input{parts/0-preamble/index}
\input{parts/1-intro/index}
\input{parts/2-theorie/index}
\input{parts/3-server/index}
\input{parts/5-mobile/index}
\input{parts/7-appendix/index}
\input{parts/0-preamble/index}
\input{parts/1-intro/index}
\input{parts/2-theorie/index}
\input{parts/3-server/index}
\input{parts/5-mobile/index}
\input{parts/7-appendix/index}
\input{parts/0-preamble/index}
\input{parts/1-intro/index}
\input{parts/2-theorie/index}
\input{parts/3-server/index}
\input{parts/5-mobile/index}
\input{parts/7-appendix/index}
\input{parts/0-preamble/index}
\input{parts/1-intro/index}
\input{parts/2-theorie/index}
\input{parts/3-server/index}
\input{parts/5-mobile/index}
\input{parts/7-appendix/index}
\input{parts/0-preamble/index}
\input{parts/1-intro/index}
\input{parts/2-theorie/index}
\input{parts/3-server/index}
\input{parts/5-mobile/index}
\input{parts/7-appendix/index}
\input{parts/0-preamble/index}
\input{parts/1-intro/index}
\input{parts/2-theorie/index}
\input{parts/3-server/index}
\input{parts/5-mobile/index}
\input{parts/7-appendix/index}
\input{parts/0-preamble/index}
\input{parts/1-intro/index}
\input{parts/2-theorie/index}
\input{parts/3-server/index}
\input{parts/5-mobile/index}
\input{parts/7-appendix/index}
\input{parts/0-preamble/index}
\input{parts/1-intro/index}
\input{parts/2-theorie/index}
\input{parts/3-server/index}
\input{parts/5-mobile/index}
\input{parts/7-appendix/index}
\input{parts/0-preamble/index}
\input{parts/1-intro/index}
\input{parts/2-theorie/index}
\input{parts/3-server/index}
\input{parts/5-mobile/index}
\input{parts/7-appendix/index}
\input{parts/0-preamble/index}
\input{parts/1-intro/index}
\input{parts/2-theorie/index}
\input{parts/3-server/index}
\input{parts/5-mobile/index}
\input{parts/7-appendix/index}
\input{parts/0-preamble/index}
\input{parts/1-intro/index}
\input{parts/2-theorie/index}
\input{parts/3-server/index}
\input{parts/5-mobile/index}
\input{parts/7-appendix/index}
\input{parts/0-preamble/index}
\input{parts/1-intro/index}
\input{parts/2-theorie/index}
\input{parts/3-server/index}
\input{parts/5-mobile/index}
\input{parts/7-appendix/index}
\input{parts/0-preamble/index}
\input{parts/1-intro/index}
\input{parts/2-theorie/index}
\input{parts/3-server/index}
\input{parts/5-mobile/index}
\input{parts/7-appendix/index}
\input{parts/0-preamble/index}
\input{parts/1-intro/index}
\input{parts/2-theorie/index}
\input{parts/3-server/index}
\input{parts/5-mobile/index}
\input{parts/7-appendix/index}
\input{parts/0-preamble/index}
\input{parts/1-intro/index}
\input{parts/2-theorie/index}
\input{parts/3-server/index}
\input{parts/5-mobile/index}
\input{parts/7-appendix/index}
\input{parts/0-preamble/index}
\input{parts/1-intro/index}
\input{parts/2-theorie/index}
\input{parts/3-server/index}
\input{parts/5-mobile/index}
\input{parts/7-appendix/index}
\input{parts/0-preamble/index}
\input{parts/1-intro/index}
\input{parts/2-theorie/index}
\input{parts/3-server/index}
\input{parts/5-mobile/index}
\input{parts/7-appendix/index}
\input{parts/0-preamble/index}
\input{parts/1-intro/index}
\input{parts/2-theorie/index}
\input{parts/3-server/index}
\input{parts/5-mobile/index}
\input{parts/7-appendix/index}
\input{parts/0-preamble/index}
\input{parts/1-intro/index}
\input{parts/2-theorie/index}
\input{parts/3-server/index}
\input{parts/5-mobile/index}
\input{parts/7-appendix/index}
\input{parts/0-preamble/index}
\input{parts/1-intro/index}
\input{parts/2-theorie/index}
\input{parts/3-server/index}
\input{parts/5-mobile/index}
\input{parts/7-appendix/index}
\input{parts/0-preamble/index}
\input{parts/1-intro/index}
\input{parts/2-theorie/index}
\input{parts/3-server/index}
\input{parts/5-mobile/index}
\input{parts/7-appendix/index}
\input{parts/0-preamble/index}
\input{parts/1-intro/index}
\input{parts/2-theorie/index}
\input{parts/3-server/index}
\input{parts/5-mobile/index}
\input{parts/7-appendix/index}
\input{parts/0-preamble/index}
\input{parts/1-intro/index}
\input{parts/2-theorie/index}
\input{parts/3-server/index}
\input{parts/5-mobile/index}
\input{parts/7-appendix/index}
\input{parts/0-preamble/index}
\input{parts/1-intro/index}
\input{parts/2-theorie/index}
\input{parts/3-server/index}
\input{parts/5-mobile/index}
\input{parts/7-appendix/index}
\input{parts/0-preamble/index}
\input{parts/1-intro/index}
\input{parts/2-theorie/index}
\input{parts/3-server/index}
\input{parts/5-mobile/index}
\input{parts/7-appendix/index}
\input{parts/0-preamble/index}
\input{parts/1-intro/index}
\input{parts/2-theorie/index}
\input{parts/3-server/index}
\input{parts/5-mobile/index}
\input{parts/7-appendix/index}
\input{parts/0-preamble/index}
\input{parts/1-intro/index}
\input{parts/2-theorie/index}
\input{parts/3-server/index}
\input{parts/5-mobile/index}
\input{parts/7-appendix/index}
\input{parts/0-preamble/index}
\input{parts/1-intro/index}
\input{parts/2-theorie/index}
\input{parts/3-server/index}
\input{parts/5-mobile/index}
\input{parts/7-appendix/index}
\input{parts/0-preamble/index}
\input{parts/1-intro/index}
\input{parts/2-theorie/index}
\input{parts/3-server/index}
\input{parts/5-mobile/index}
\input{parts/7-appendix/index}
\input{parts/0-preamble/index}
\input{parts/1-intro/index}
\input{parts/2-theorie/index}
\input{parts/3-server/index}
\input{parts/5-mobile/index}
\input{parts/7-appendix/index}
\input{parts/0-preamble/index}
\input{parts/1-intro/index}
\input{parts/2-theorie/index}
\input{parts/3-server/index}
\input{parts/5-mobile/index}
\input{parts/7-appendix/index}
\input{parts/0-preamble/index}
\input{parts/1-intro/index}
\input{parts/2-theorie/index}
\input{parts/3-server/index}
\input{parts/5-mobile/index}
\input{parts/7-appendix/index}
\input{parts/0-preamble/index}
\input{parts/1-intro/index}
\input{parts/2-theorie/index}
\input{parts/3-server/index}
\input{parts/5-mobile/index}
\input{parts/7-appendix/index}
\input{parts/0-preamble/index}
\input{parts/1-intro/index}
\input{parts/2-theorie/index}
\input{parts/3-server/index}
\input{parts/5-mobile/index}
\input{parts/7-appendix/index}

\newpage

\input{parts/0-preamble/index}
\input{parts/1-intro/index}
\input{parts/2-theorie/index}
\input{parts/3-server/index}
\input{parts/5-mobile/index}
\input{parts/7-appendix/index}
\input{parts/0-preamble/index}
\input{parts/1-intro/index}
\input{parts/2-theorie/index}
\input{parts/3-server/index}
\input{parts/5-mobile/index}
\input{parts/7-appendix/index}
\input{parts/0-preamble/index}
\input{parts/1-intro/index}
\input{parts/2-theorie/index}
\input{parts/3-server/index}
\input{parts/5-mobile/index}
\input{parts/7-appendix/index}
\input{parts/0-preamble/index}
\input{parts/1-intro/index}
\input{parts/2-theorie/index}
\input{parts/3-server/index}
\input{parts/5-mobile/index}
\input{parts/7-appendix/index}
\input{parts/0-preamble/index}
\input{parts/1-intro/index}
\input{parts/2-theorie/index}
\input{parts/3-server/index}
\input{parts/5-mobile/index}
\input{parts/7-appendix/index}
\input{parts/0-preamble/index}
\input{parts/1-intro/index}
\input{parts/2-theorie/index}
\input{parts/3-server/index}
\input{parts/5-mobile/index}
\input{parts/7-appendix/index}
\input{parts/0-preamble/index}
\input{parts/1-intro/index}
\input{parts/2-theorie/index}
\input{parts/3-server/index}
\input{parts/5-mobile/index}
\input{parts/7-appendix/index}
\input{parts/0-preamble/index}
\input{parts/1-intro/index}
\input{parts/2-theorie/index}
\input{parts/3-server/index}
\input{parts/5-mobile/index}
\input{parts/7-appendix/index}
\input{parts/0-preamble/index}
\input{parts/1-intro/index}
\input{parts/2-theorie/index}
\input{parts/3-server/index}
\input{parts/5-mobile/index}
\input{parts/7-appendix/index}
\input{parts/0-preamble/index}
\input{parts/1-intro/index}
\input{parts/2-theorie/index}
\input{parts/3-server/index}
\input{parts/5-mobile/index}
\input{parts/7-appendix/index}
\input{parts/0-preamble/index}
\input{parts/1-intro/index}
\input{parts/2-theorie/index}
\input{parts/3-server/index}
\input{parts/5-mobile/index}
\input{parts/7-appendix/index}
\input{parts/0-preamble/index}
\input{parts/1-intro/index}
\input{parts/2-theorie/index}
\input{parts/3-server/index}
\input{parts/5-mobile/index}
\input{parts/7-appendix/index}
\input{parts/0-preamble/index}
\input{parts/1-intro/index}
\input{parts/2-theorie/index}
\input{parts/3-server/index}
\input{parts/5-mobile/index}
\input{parts/7-appendix/index}
\input{parts/0-preamble/index}
\input{parts/1-intro/index}
\input{parts/2-theorie/index}
\input{parts/3-server/index}
\input{parts/5-mobile/index}
\input{parts/7-appendix/index}
\input{parts/0-preamble/index}
\input{parts/1-intro/index}
\input{parts/2-theorie/index}
\input{parts/3-server/index}
\input{parts/5-mobile/index}
\input{parts/7-appendix/index}
\input{parts/0-preamble/index}
\input{parts/1-intro/index}
\input{parts/2-theorie/index}
\input{parts/3-server/index}
\input{parts/5-mobile/index}
\input{parts/7-appendix/index}
\input{parts/0-preamble/index}
\input{parts/1-intro/index}
\input{parts/2-theorie/index}
\input{parts/3-server/index}
\input{parts/5-mobile/index}
\input{parts/7-appendix/index}
\input{parts/0-preamble/index}
\input{parts/1-intro/index}
\input{parts/2-theorie/index}
\input{parts/3-server/index}
\input{parts/5-mobile/index}
\input{parts/7-appendix/index}
\input{parts/0-preamble/index}
\input{parts/1-intro/index}
\input{parts/2-theorie/index}
\input{parts/3-server/index}
\input{parts/5-mobile/index}
\input{parts/7-appendix/index}
\input{parts/0-preamble/index}
\input{parts/1-intro/index}
\input{parts/2-theorie/index}
\input{parts/3-server/index}
\input{parts/5-mobile/index}
\input{parts/7-appendix/index}
\input{parts/0-preamble/index}
\input{parts/1-intro/index}
\input{parts/2-theorie/index}
\input{parts/3-server/index}
\input{parts/5-mobile/index}
\input{parts/7-appendix/index}
\input{parts/0-preamble/index}
\input{parts/1-intro/index}
\input{parts/2-theorie/index}
\input{parts/3-server/index}
\input{parts/5-mobile/index}
\input{parts/7-appendix/index}
\input{parts/0-preamble/index}
\input{parts/1-intro/index}
\input{parts/2-theorie/index}
\input{parts/3-server/index}
\input{parts/5-mobile/index}
\input{parts/7-appendix/index}
\input{parts/0-preamble/index}
\input{parts/1-intro/index}
\input{parts/2-theorie/index}
\input{parts/3-server/index}
\input{parts/5-mobile/index}
\input{parts/7-appendix/index}
\input{parts/0-preamble/index}
\input{parts/1-intro/index}
\input{parts/2-theorie/index}
\input{parts/3-server/index}
\input{parts/5-mobile/index}
\input{parts/7-appendix/index}
\input{parts/0-preamble/index}
\input{parts/1-intro/index}
\input{parts/2-theorie/index}
\input{parts/3-server/index}
\input{parts/5-mobile/index}
\input{parts/7-appendix/index}
\input{parts/0-preamble/index}
\input{parts/1-intro/index}
\input{parts/2-theorie/index}
\input{parts/3-server/index}
\input{parts/5-mobile/index}
\input{parts/7-appendix/index}
\input{parts/0-preamble/index}
\input{parts/1-intro/index}
\input{parts/2-theorie/index}
\input{parts/3-server/index}
\input{parts/5-mobile/index}
\input{parts/7-appendix/index}
\input{parts/0-preamble/index}
\input{parts/1-intro/index}
\input{parts/2-theorie/index}
\input{parts/3-server/index}
\input{parts/5-mobile/index}
\input{parts/7-appendix/index}
\input{parts/0-preamble/index}
\input{parts/1-intro/index}
\input{parts/2-theorie/index}
\input{parts/3-server/index}
\input{parts/5-mobile/index}
\input{parts/7-appendix/index}
\input{parts/0-preamble/index}
\input{parts/1-intro/index}
\input{parts/2-theorie/index}
\input{parts/3-server/index}
\input{parts/5-mobile/index}
\input{parts/7-appendix/index}
\input{parts/0-preamble/index}
\input{parts/1-intro/index}
\input{parts/2-theorie/index}
\input{parts/3-server/index}
\input{parts/5-mobile/index}
\input{parts/7-appendix/index}
\input{parts/0-preamble/index}
\input{parts/1-intro/index}
\input{parts/2-theorie/index}
\input{parts/3-server/index}
\input{parts/5-mobile/index}
\input{parts/7-appendix/index}
\input{parts/0-preamble/index}
\input{parts/1-intro/index}
\input{parts/2-theorie/index}
\input{parts/3-server/index}
\input{parts/5-mobile/index}
\input{parts/7-appendix/index}
\input{parts/0-preamble/index}
\input{parts/1-intro/index}
\input{parts/2-theorie/index}
\input{parts/3-server/index}
\input{parts/5-mobile/index}
\input{parts/7-appendix/index}
\input{parts/0-preamble/index}
\input{parts/1-intro/index}
\input{parts/2-theorie/index}
\input{parts/3-server/index}
\input{parts/5-mobile/index}
\input{parts/7-appendix/index}