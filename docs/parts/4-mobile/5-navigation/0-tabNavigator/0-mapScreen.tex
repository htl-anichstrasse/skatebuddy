\subsection{MapScreen}
Die Umsetzung der Kartenfunktion, um alle Skateparks anzuzeigen, war das zweit-komplizierteste
Feature in der App. Die Bibliothek \textbf{react-native-maps} ist hierbei die Basis dieser Funktion.

\begin{code}[htp]
\begin{lstlisting}[firstnumber=1,language=JavaScript, style=JSX]
const MapScreen = ({ navigation }) => {
  const {
    data: skateparks,
    isLoading,
    error,
    refreshData,
  } = useFetch('https://skate-buddy.josholaus.com/api/skateparks');
  const mapRef = useRef(null);

  return (
    <View style={styles.container}>
      {isLoading && <LoadingCircle />}
      {error && <Error error={error} refresh={refreshData} />}
      {skateparks && (
        <>
          <Map
            mapRef={mapRef}
            skateparks={skateparks}
            navigation={navigation}
          />
          <CircleButton
            onPress={() => {
              mapRef.current.animateCamera({
                center: {
                  latitude: 47.27,
                  longitude: 11.4,
                },
                altitude: 1000,
                pitch: 0,
                heading: 0,
                zoom: 12,
              });
            }}
          />
        </>
      )}
    </View>
  );
};
export default MapScreen;
\end{lstlisting}
\caption{React Component - Der Karten-Tab}
\end{code}

Zuerst werden mit dem Hook \textbf{useFetch} alle Skateparks von der API abgefragt, solange die Variable
\textbf{skateparks} \textbf{null} ist, wird nur ein Ladesymbol angezeigt. Anschließend wird die Map erzeugt, wie auch
ein Knopf, mit dem man die Position der Karte auf Innsbruck zentrieren kann.

\begin{figure}[H]
  \begin{center}
    \includegraphics[width=0.6\textwidth]{Mobile/Map/MapScreen.png}
    \caption{Alle Skateparks werden an der korrekten Stelle angezeigt}
  \end{center}
\end{figure}

\newpage
\subsubsection{Map}

\begin{code}[htp]
\begin{lstlisting}[firstnumber=1,language=JavaScript, style=JSX]
const Map = ({ skateparks, mapRef, navigation }) => {
  return (
    <MapView
      // props
      showsUserLocation
      provider={PROVIDER_GOOGLE}
      style={mapStyles.map}
      ref={mapRef}
      initialCamera={camera}
      showsPointsOfInterest={false}
      showsCompass={false}
      showsIndoors={false}
      minZoomLevel={8}
      rotateEnabled={false}
      pitchEnabled={false}
      mapType="satellite"
    >
      <SkateparkMarkers
        skateparks={skateparks}
        mapRef={mapRef}
        navigation={navigation}
      />
    </MapView>
  );
};
export default Map;
\end{lstlisting}
\caption{React Component - Die MapView-Komponente aus react-native-maps}
\end{code}

In der Komponente \textbf{Map} wird einfach eine \textbf{MapView} aufgerufen, welche die Google Maps Ansicht
beinhaltet. Die Parks werden anschließend wieder an die nächste Komponente weitergegeben, wo sie in
die Map-Marker umgewandelt werden. Drückt man auf einen Marker, so wird der Park zentriert und ein
Textfeld wird angezeigt. Wird dies angeklickt, so wird man zu den Details vom jeweiligen Park
weitergeleitet.

\begin{figure}[H]
  \begin{center}
    \includegraphics[width=0.6\textwidth]{Mobile/Map/MapMarkerFocused.png}
    \caption{Ein ausgewählter Marker}
  \end{center}
\end{figure}
