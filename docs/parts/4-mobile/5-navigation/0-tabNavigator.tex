\section{Tab-Navigation}
Andere mit React Native gebauten Apps, wie Facebook oder Instagram, verwenden eine sogenannte
"Bottom-Tabs"-Navigation, also eine Tab-Basierte Navigation auf der Unterseite des Bildschirms, um
ihre Hauptfunktionen dem Benutzer zu präsentieren.

Um die App simpel zu halten ist dies auch unsere präferierte Navigations-Methode, um die einzelnen
Bereiche der Anwendung miteinander zu verbinden.

\begin{lstlisting}
import MapScreen from '../screens/MapScreen';
import ProfileScreen from '../screens/ProfileScreen';
import SkateparksStack from './SkateparksStack';
import Colors from '../styles/Colors';

const Tab = createBottomTabNavigator();

const BottomTabsNavigator = () => {
  const tabBarIcons = (route, focused, color) => {
    ...
  };

  return (
    <Tab.Navigator
      backBehavior="initialRoute"
      initialRouteName="Skateparks"
      screenOptions={({ route }) => ({
        tabBarIcon: ({ focused, color }) => tabBarIcons(route, focused, color),
        tabBarActiveTintColor: Colors.primary,
        tabBarInactiveTintColor: Colors.gray2,
        tabBarActiveBackgroundColor: Colors.primarySoft,
        tabBarShowLabel: false,
        headerShown: false,
      })}
    >
      <Tab.Screen name="Skateparks" component={SkateparksStack} />
      <Tab.Screen name="Map" component={MapScreen} />
      <Tab.Screen name="Profile" component={ProfileScreen} />
    </Tab.Navigator>
  );
};

export default BottomTabsNavigator;
\end{lstlisting}

