\section{useFetch}
Der useFetch-Hook wurde bereits in der Webseite verwendet, die App nutzt eine leicht abgewandelte
Version, damit man auch die Daten aktualisieren kann. Mehr dazu im Abschnitt \nameref{getData}.

\begin{lstlisting}
  ...
  const [refresher, setRefresher] = useState(false);

  const refreshData = () => {
    setRefresher(prevRefresher => !prevRefresher);
  };

  useEffect(() => {
    ...
  }, [url, refresher]);
  ...
\end{lstlisting}

useEffect ist ein React-Hook. Er führt seine übergebene Funktion dann aus, wenn sich eine Variable
in der im zweiten Argument übergebenen Liste ändert. Sollte die Liste leer sein, so wird die
Funktion nur ausgeführt, sobald die Komponente das erste Mal gerendert wird.

Sobald die Variable refresher nun also geändert wird, wird die Funktion im useEffect-Hook neu
ausgeführt.