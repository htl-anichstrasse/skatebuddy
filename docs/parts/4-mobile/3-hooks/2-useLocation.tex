\section{useLocation}
In diesem Hook wird die aktuelle GPS-Position vom Gerät abgefragt. Der Benutzer wird, sofern er dies
noch nicht getan hat, nach der Erlaubnis gefragt, die genaue Position des Gerätes zu ermitteln und
mit der App zu teilen. Dies wird benötigt, um die Distanz zwischen Benutzer und Skateparks zu
berechnen.

\begin{lstlisting}
const checkPermission = async () => {
  if (Platform.OS === 'android') {
    const status = await PermissionsAndroid.check(
      PermissionsAndroid.PERMISSIONS.ACCESS_FINE_LOCATION,
    );

    if (status) {
      setLocError(null);
      return true;
    }
  } else if (Platform.OS === 'ios') {
    // TODO Implement iOS permission check
  }
  return false;
};
\end{lstlisting}

In Zeile 3 wird mit Hilfe der PermissionsAndroid-API, bereitgestellt von React Native, überprüft
ob die Erlaubnis schon erteilt wurde, die Position zu ermitteln.

Sollte dies nicht der Fall sein, so wird die Erlaubnis angefragt.

\begin{lstlisting}
const requestPermission = async () => {
  if (Platform.OS === 'android') {
    const granted = await PermissionsAndroid.request(
      PermissionsAndroid.PERMISSIONS.ACCESS_FINE_LOCATION,
    );
    if (granted === PermissionsAndroid.RESULTS.GRANTED) {
      setLocError(null);
      return true;
    }
  } else if (Platform.OS === 'ios') {
    // TODO: Implement iOS permission request
  }
  setLocError('Standortdienst wurde abgelehnt');
  return false;
};
\end{lstlisting}

\begin{lstlisting}
const getLocation = async () => {
  setLocLoading(true);
  setLocError(null);
  if ((await checkPermission()) || (await requestPermission())) {
    Geolocation.getCurrentPosition(
      geolocation => {
        setLocation(geolocation);
        setLocLoading(false);
      },
      error => {
        setLocError(`${error.code} ${error.message}`);
        setLocLoading(false);
      },
      { enableHighAccuracy: true, timeout: 15000, maximumAge: 10000 },
    );
  }
};
\end{lstlisting}

Diese Funktion wird jedes Mal aufgerufen, um die Position zu bestimmen. Zeile 4 zeigt, dass zuerst
geprüft wird, ob die Erlaubnis schon erteilt wurde. Falls nicht wird sie angefragt. Falls dies
funktioniert haben sollte, wird in Zeile 5 die Position ausgelesen und, in Zeile 7, abgespeichert.
Geolocation ist in der Bibliothek "react-native-geolocation-service" vorhanden.