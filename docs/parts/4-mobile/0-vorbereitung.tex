\chapter{Vorbereitung}

\section{Warum habe ich mich für React Native entschieden?}
Im Sommer 2021 habe ich während meiner Arbeitszeit im GRZ IT Center in Innsbruck dazu entschieden,
jeden Abend ein paar Stunden in das Lernen von neuen Technologien in Sachen Webentwicklung zu
investieren. Als erstes auf meiner Liste stand der Kurs Modern JavaScript von NetNinja auf YouTube.
NetNinja bietet auf seiner Webseite zu jedem freien Youtube-Kurs einen noch detaillierteren Kurs an,
jedoch wollte ich kein Geld ausgeben, um zu lernen und offensichtlich macht es dann doch keinen
großen Unterschied. Nach Modern JavaScript war klar, dass ich nun bereit war ein JavaScript-
Framework zu lernen, um Benutzeroberflächen zu gestalten. \\

Durch Youtube hörte ich oft von den drei Frameworks React von Facebook, Angular von Google und Vue
vom ehemaligen Google-Mitarbeiter Evan You. Ich wählte React aus, da es in Österreich und generell
weltweit die meiste Anzahl von Jobs anbot. Nachdem ich React gelernt hatte und feststand, dass wir
für unsere Diplomarbeit eine App benötigen, schlug ich sofort React Native als Cross-Platform-Lösung
vor, damit ich mein bereits angeeignetes Wissen direkt anwenden kann. \\

Am 30. November 2021 begann ich offiziell mit der Entwicklung unserer Skater-App. Ich nutzte
zeitweise das Projektmanagement-Tool Trello, um den Überblick über das Projekt zu behalten, es ist
schließlich das bisher größte Projekt, an dem wir drei je gearbeitet haben. Im Nachhinein denke ich,
es wäre viel besser gewesen, wenn ich sofort alles dokumentiert hätte und nicht jetzt, vier Monate
später, mir alles noch einmal anschauen müsste.