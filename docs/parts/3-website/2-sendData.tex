\section{Daten senden}
\label{sendData}

Natürlich ist es auch möglich Daten über die Webseite an den Server senden. Dies wird beim 
Erstellen eines Benutzers als sowohl auch beim einloggen und beim Erstellen eines Parks oder einer 
Review zu einem Park benötigt. Das Senden der Daten an den Server erfolgt ebenfalls mit einem \textit{Fetch},
diesmal jedoch ist der fetch keine \textit{get Method} wie bei dem useFetch, sondern eine 
\textit{post Method}.
\begin{lstlisting}
    fetch("APU_URL", {
        method: 'POST',
        headers: {"Content-Type": "application/json"},
        body: JSON.stringify(data)
          },
        ).then(function(response){
          return response.json();
        })
\end{lstlisting}
Im header wird angegeben, welcher Datentyp (In unserem Fall ein JSON) gesendet wird. Im body wird der 
Datenstring in ein JSON umgewandelt und anschließend an die entsprechende URL gesendet. Dann wird auf 
eine Antwort vom Server gewartet und danach wird diese Antwort als JSON returned. 
Diese response können wir dann nach beliebigen verwenden.