\section{Daten senden}
\label{sendData}

Natürlich kann man über die Webseite auch Daten an den Server senden. Dies wird benötigt beim 
Erstellen eines Benutzers sowohl als auch beim einloggen und beim Erstellen eines Parks oder einer 
Review zu einem Park. Das Senden der Daten an der Server erfolgt ebenfalls mit einem \textit{Fetch}
diesmal jedoch ist der fetch keine \textit{get Method} wie bei dem useFetch, sondern eine 
\textit{post Method}.
\begin{lstlisting}
    fetch("APU_URL", {
        method: 'POST',
        headers: {"Content-Type": "application/json"},
        body: JSON.stringify(data)
          },
        ).then(function(response){
          return response.json();
        })
\end{lstlisting}

Im header geben wir an, welchen Datentyp (In unserem Fall ein JSON) wir senden. Im body wandeln wir 
unseren Datenstring in ein JSON um und senden diesen dann an die entsprechende URL. Dann warten wir 
auf eine Antwort vom Server, und returnen diese als json. Diese response können wir dann nach beliebigen 
verwenden.