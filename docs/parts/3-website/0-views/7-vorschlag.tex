\section{Vorschläge erstellen}
\label{vorschläge}

\begin{figure}[H]
    \begin{center}
      \frame{\includegraphics[width=1\textwidth]{Website/Vorschläge.png}}
      \caption{Einen Vorschlag erstellen}
    \end{center}
\end{figure}

Einen Vorschlag erstellen kann jeder eingeloggte Benutzer. Diese Vorschläge können später von den 
Admins abgelehnt werden oder als Park hinzugefügt werden. Wie auch schon der Login oder das Account 
erstellen funktioniert auch das Vorschläge erstellen über ein HTML-Formular. Der Latitude muss sich 
jedoch innerhalb von +- 90 befinden und der Longitude innerhalb von +-180. Die Hindernisse sind
ein HTML-select welche bei Betätigung des Knopfes das Hindernis als sowohl auch die Schwierigkeit 
in eine Liste gespeichert. Diese Liste wird mittels map Funktion als HTML-Tabelle ausgegeben. Die 
Elemente dieser Liste erhalten eine temporäre Id. Diese temporäre Id wird beim \textit{Entfernen} 
Knopf verwendet, da dieser das Element anhand der Id entfernt. 