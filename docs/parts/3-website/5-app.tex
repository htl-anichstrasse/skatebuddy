\section{App.js}
\label{appDatei}

In der App Komponente ist die \nameref{nav} und der \nameref{footer} definiert, 
also sowohl auch die URLs zu den einzigen \nameref{views}. Diese URLs sind mittels einen React 
Router definiert. Da der Footer direkt in der App-Komponente ausprogrammiert ist, 
befindet sich die CSS auch in der CSS-Datei der App-Komponente. Die Navigationsleiste wird genau 
wie die Views als Komponente reingeladen.

\subsection{React Router, Routes, Route}
Alle Komponenten aus dem die Webseite besteht sind innerhalb einer Router Komponente definiert. Die 
Router Komponente sorgt dafür, dass die Seite wie eine \nameref{singlePageApp} funktioniert. Betätigt
der Benutzer einen Wechsel der View, was normalerweise einen Seitenwechsel verursachen würde, wird die 
Anfrage an den Server vom React Router abgefangen und der Router sendet die neue View ohne einen 
Seitenwechsel zu verursachen. 


Innerhalb der Router-Komponente, nämlich dort wo die View angezeigt werden soll (in meinem Fall unter
der Navigationsleiste) ist die Routes Komponente definiert.
\begin{lstlisting}
    <Routes>
        <Route path="/" element={<Home />} />
        <Route path="/parks" element={<Parks />} />
        <Route path="/skateparks/:id" element={<ParkDetails />} />
        <Route path="/LogIn" element={<LogIn />}/>
        <Route path="/CreateAccount" element={<CreateAccount/>}/>
        <Route path="/AllMap" element={<AllMap/>}/>
        <Route path="/Profile" element={<Profile/>}/>
        <Route path="/AddPark" element={<AddPark/>}/>
        <Route path="/AddRecommendation" element={<AddRecommendation/>}/>
        <Route path="/Recommendations" element={<Recommendationlist/>}/>
    </Routes>
\end{lstlisting} 
Innerhalb der Routes Komponente werden die einzelnen Routen definiert. Je nachdem welche URL aufgerufen wird,
wird eine andere Komponente geladen. Diese Routen werden mittels \nameref{React-Links} in den einzelnen 
Views geändert. Zum Beispiel wird bei der URL \textit{/parks} die Komponente \textbf{Parks} als 
Inhalt geladen. Wie man sieht ist bei der \textit{/skateparks} URL ein /:id am Ende. Dieses :id 
bedeutet, dass je nach welcher id man in die URL schreibt also 1, 2, 3, ... man der 
ParkDetails-Komponente einen anderen Parameter übergibt. 

\subsubsection{Link-Komponenten}
\label{React-Links}
Die Link-Komponente dient dazu die URL zu ändern falls man einen Viewwechsel durchführen möchte.
\begin{lstlisting}
   <Link to="/" className='home-nav' id="nav">
       Home
   </Link>
\end{lstlisting}
Durch das Klicken auf das Wort \textit{Home} wird die URL in diesem Beispiel auf \textit{/} geändert.
Mit dieser geänderten URL ändern der Router mit den definierten Routes die View.

\subsection{Navigationsleiste}
\label{nav}

Die Navigationsleiste ist genauso wie der \nameref{footer} immer sichtbar. Sie steht über dem Inhalt der Seite und 
dient zur Navigation der Webseite. Die Navigation erfolgt ganz einfach über \nameref{React-Links}.

\begin{figure}[H]
  \begin{center}
    \frame{\includegraphics[width=1\textwidth]{Website/Not-Logged-In-Navbar.png}}
    \caption{Navbar der Webseite wenn man nicht eingeloggt ist}
  \end{center}
\end{figure}

Das Design der Navigationsleiste ist wie auch der Rest der Seite sehr simpel gehalten. Schwebt man 
mit der Maus über eins der Felder mit denen man die View wechseln kann, wird dieses unterstrichen.


Klickt man auf Home, Parks, Login oder Account Erstellen wird die Link-Komponente welche dich zu den 
jeweiligen Views bringt aufgerufen. Ist man auf der Webseite jedoch eingeloggt, verschwinden jedoch 
die Login und Account erstellen Felder und es erscheint ein Profil Feld.

\begin{figure}[H]
    \begin{center}
      \frame{\includegraphics[width=1\textwidth]{Website/Logged-In-Navbar.png}}
      \caption{Navbar der Webseite wenn man eingeloggt ist}
    \end{center}
\end{figure}

\pagebreak

Das Überprüfen ob man eingeloggt ist oder nicht erfolgt ganz einfach über \nameref{jsx}.


\begin{lstlisting}
    {token && 
        <Link to="/Profile" className='pofile-nav' id="nav">
          Profile
        </Link>
    }

    {!token &&
        <>
         <Link to="/LogIn" className='login-nav' id="nav">
          Login
        </Link>
        <Link to="/CreateAccount" className='createAccount-nav' id="nav">
              Account Erstellen
            </Link>
        </>
      }
\end{lstlisting}

Ist die token Variable nicht \textit{null} bzw nicht \textit{false}, wird der Code in den geschwungenen Klammern 
mit token ausgeführt. Ist die Variable \textit{null oder false} wird der Code mit dem !token ausgeführt. 
Die Variable token beinhaltet dies, was in dem SessionStorage der Seite steht. Ist man also eingeloggt, ist diese 
nicht \textit{null} und der Code mit dem Profile-Link wird ausgeführt. Ist man nicht eingeloggt ist token 
\textit{null} und der Code in den Klammern mit !token wird ausgeführt.

\subsection{Fußzeile}
\label{footer}

Die Fußzeile ist ein simpler Absatz mit Information am Ende des Inhaltes.
Sie ist wie auch die Navigationsleiste auf jeder View die selbe. In der Fußzeile befindet sich neben 
der Information auch ein Top-Knopf, welcher bei Betätigung ganz nach oben scrollt. 

\begin{figure}[H]
    \begin{center}
      \frame{\includegraphics[width=0.8\textwidth]{Website/Footer.png}}
      \caption{Die Fußzeile der Webseite}
    \end{center}
\end{figure}

\pagebreak

Damit der Knopf bei Mausklick von selbst nach oben scrollt, weiße ich ihm eine JavaScript-Funktion bei
Knopfdruck zu.

\begin{lstlisting}
    function topFunction() {
        document.body.scrollTop = 0;
        document.documentElement.scrollTop = 0;
    }

    <button onClick={topFunction} id="myBtn" title="Go to top" className="top-button">Top</button>
\end{lstlisting}

Bei Knopfdruck wird diese simple Funktion ausgeführt, welche ganz nach oben scrollen. Damit die 
Funktion scrollt und nicht auf einmal ganz oben ist, musste noch folgendes in der CSS Datei 
hinzugefügt werde.

\begin{lstlisting}
    html {
        scroll-behavior: smooth;
    }
\end{lstlisting}

Damit springt die Seite mit Knopfdruck nicht nach oben, sondern scrollt reibungslos.