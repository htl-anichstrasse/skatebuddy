\section{App.js}
\label{appDatei}

In der App Komponente ist die \underline{\nameref{nav}} und der \underline{\nameref{footer}} definiert, 
als sowohl auch die URLs zu den einzelnen \underline{\nameref{views}}. Diese URLs sind mit einen React 
Router definiert. Da der Footer direkt in der App-Komponente ausprogrammiert ist, 
befindet sich die CSS des footers auch in der CSS-Datei der App-Komponente. Die Navigationsleiste wird genau 
wie die Views als Komponente reingeladen.

\subsection{React Router, Routes, Route}
Alle Komponenten aus denen die Webseite besteht sind innerhalb einer Router Komponente definiert. Die 
Router Komponente sorgt dafür, dass die Seite wie eine \underline{\nameref{singlePageApp}} funktioniert. Betätigt
der Benutzer einen Wechsel der View, was normalerweise einen Seitenwechsel verursachen würde, wird die 
Anfrage an den Server vom React Router abgefangen und der Router sendet die neue View ohne einen 
Seitenwechsel zu verursachen. 


Innerhalb der Router-Komponente, nämlich dort wo die View angezeigt werden soll, ist die Routes
Komponente definiert.
\begin{code}[htp]
\begin{lstlisting}
    <Routes>
        <Route path="/" element={<Home />} />
        <Route path="/parks" element={<Parks />} />
        <Route path="/skateparks/:id" element={<ParkDetails />} />
        <Route path="/LogIn" element={<LogIn />}/>
        <Route path="/CreateAccount" element={<CreateAccount/>}/>
        <Route path="/AllMap" element={<AllMap/>}/>
        <Route path="/Profile" element={<Profile/>}/>
        <Route path="/AddPark" element={<AddPark/>}/>
        <Route path="/AddRecommendation" element={<AddRecommendation/>}/>
        <Route path="/Recommendations" element={<Recommendationlist/>}/>
    </Routes>
\end{lstlisting} 
\caption{React Component - Die Routen der Webseite}
\end{code}
Innerhalb der Routes Komponente werden die einzelnen Routen definiert. Je nachdem welche URL aufgerufen wird,
wird eine andere Komponente geladen. Diese Routen werden mittels \underline{\nameref{React-Links}} in den einzelnen 
Views geändert. Zum Beispiel wird bei der URL \textit{/parks} die Komponente \textbf{Parks} als 
Inhalt geladen. Bei der \textit{/skateparks} URL ist ein /:id am Ende. Dieses :id 
bedeutet, dass je nach welche id in die URL geschrieben wird, also 1, 2, 3, ... der 
ParkDetails-Komponente ein anderer Parameter übergeben wird. 

\subsubsection{Link-Komponenten}
\label{React-Links}
Die Link-Komponente dient dazu die URL zu ändern falls ein Viewwechsel durchgeführt werden soll.
\begin{code}[htp]
\begin{lstlisting}
   <Link to="/" className='home-nav' id="nav">
       Home
   </Link>
\end{lstlisting}
\caption{React Component - React Link Komponente}
\end{code}
Durch das Klicken auf das Wort \textit{Home} wird die URL in diesem Beispiel auf \textit{/} geändert.
Mit dieser geänderten URL ändert der Router mit den definierten Routes die View.

\subsection{Navigationsleiste}
\label{nav}

Die Navigationsleiste ist genauso wie der \underline{\nameref{footer}} immer sichtbar. Sie steht über dem Inhalt der Seite und 
dient zur Navigation. Die Navigation erfolgt ganz einfach über \underline{\nameref{React-Links}}.

\begin{figure}[H]
  \begin{center}
    \frame{\includegraphics[width=1\textwidth]{Website/Not-Logged-In-Navbar.png}}
    \caption{Navbar der Webseite wenn man nicht eingeloggt ist}
  \end{center}
\end{figure}

Das Design der Navigationsleiste ist wie auch der Rest der Seite sehr simpel gehalten. Befindet sich
die Maus über eins der Felder mit denen sich die View wechseln lässt, wird dieses unterstrichen.


Ist der Benutzer auf der Webseite jedoch eingeloggt, verschwinden die Login und Account erstellen 
Felder und es erscheint ein Profil Feld.

\begin{figure}[H]
    \begin{center}
      \frame{\includegraphics[width=1\textwidth]{Website/Logged-In-Navbar.png}}
      \caption{Navbar der Webseite wenn der Benutzer eingeloggt ist}
    \end{center}
\end{figure}

\pagebreak

Das Überprüfen ob der Benutzer eingeloggt ist oder nicht erfolgt ganz einfach über \underline{\nameref{jsx}}.


\begin{code}[htp]
\begin{lstlisting}
    {token && 
        <Link to="/Profile" className='pofile-nav' id="nav">
          Profile
        </Link>
    }

    {!token &&
        <>
         <Link to="/LogIn" className='login-nav' id="nav">
          Login
        </Link>
        <Link to="/CreateAccount" className='createAccount-nav' id="nav">
              Account Erstellen
            </Link>
        </>
      }
\end{lstlisting}
\caption{React Component - Überprüfen des Logins für die Navigationsleiste}
\end{code}

Ist die token Variable nicht \textit{null}, wird der Code in den geschwungenen Klammern 
mit token ausgeführt. Ist die Variable \textit{null} wird der Code mit dem !token ausgeführt. 
Die Variable token beinhaltet dies, was in dem SessionStorage der Seite steht. Ist der Verwender also eingeloggt, ist diese 
nicht \textit{null} und der Code mit dem Profile-Link wird ausgeführt. Ist der Verwender nicht eingeloggt ist token 
\textit{null} und der Code in den Klammern mit !token wird ausgeführt.

\subsection{Fußzeile}
\label{footer}

Die Fußzeile ist ein simpler Absatz mit Information am Ende des Inhaltes.
Sie ist wie auch die Navigationsleiste auf jeder View dieselbe. In der Fußzeile befindet sich neben 
der Information auch ein Top-Knopf, welcher bei Betätigung ganz nach oben scrollt. 

\begin{figure}[H]
    \begin{center}
      \frame{\includegraphics[width=0.8\textwidth]{Website/Footer.png}}
      \caption{Die Fußzeile der Webseite}
    \end{center}
\end{figure}

\pagebreak

Damit der Knopf bei Mausklick von selbst nach oben scrollt, wird ihm eine JavaScript-Funktion 
zugewiesen, welche bei Knopfdruck \textit{(onClick)} ausgeführt wird.

\begin{code}[htp]
\begin{lstlisting}
    function topFunction() {
        document.body.scrollTop = 0;
        document.documentElement.scrollTop = 0;
    }

    <button onClick={topFunction} id="myBtn" title="Go to top" className="top-button">Top</button>
\end{lstlisting}
\caption{JavaScript Funktion - Scroll top Funktion}
\end{code}

Dass die Funktion scrollt und nicht auf einmal ganz oben ist, bewirken folgende CSS-Zeilen.

\begin{code}[htp]
\begin{lstlisting}
    html {
        scroll-behavior: smooth;
    }
\end{lstlisting}
\caption{CSS - smooth scrolling}
\end{code}

Damit springt die Seite mit Knopfdruck nicht nach oben, sondern scrollt reibungslos.