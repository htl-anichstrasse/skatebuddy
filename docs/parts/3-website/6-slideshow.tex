\section{Slideshow}
\label{slideshow}

Auf der \nameref{home} und auf der \nameref{parklist} gibt es eine Slideshow von Bildern.
Diese Slideshow Komponente ist extern und habe ich mittels npm heruntergeladen und in meinem 
Projekt importiert. Der Ersteller dieser Komponente ist \textit{femioladeji}. Verwenden tu ich diese 
Komponente wie folgt:

\begin{lstlisting}
    const Slideshow = (parkpics) => {
    return (
      <div className="slide-container">
        <Slide autoplay={false} transitionDuration={420}>
         {parkpics.parkpics.map(pic=> (
            <div className="each-slide" key={pic.skateparkId}>
              <div style={{'backgroundImage': `url(https://skate-buddy.josholaus.com/api/skateparkpictures/${pic.skateparkId}/${pic.picId})` }} className='bg'>
              </div>
            </div>
          ))} 
        </Slide>
      </div>
    )
}
\end{lstlisting}

Wie man sieht übergebe ich der JavaScript Funktion eine Parameter names \textit{parkpics}. Die 
Daten für diesen Parameter hole ich mir ganz einfach mittel \textit{useFetch}. In diesem Parameter 
befindet sich ein zwei-dimensionales Array, welches die ParkID des Parks also sowohl auch die ID des 
Bildes enthält. Insgesamt übergib ich der der Slide Komponente zwei Parameter: autoplay={false} sorgt
dafür, dass die Bilder nicht von selbst weiter schalten. (Auf der \nameref{home} ist diese Variable auf true) und
eine transitionDuration welche die Geschwindigkeit festlegt, mit der die Bilder weiter geschaltet werden soll.
Innerhalb der Slide Komponente steht eine map Funktion, welche für jedes Element im Parameter parkpics 
ein div erzeugt, welche das jeweilige Bild als Hintergrundbild besitzt. 


