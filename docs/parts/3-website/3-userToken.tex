\section{Benutzertoken}
\label{benutzer}

Der Benutzer kann sich mittels eines einfachen Formular in die Webseite einloggen als auch einen neuen 
Account erstellen. Bei erfolgreichem \nameref{login} oder \nameref{createAccount} erhält man vom Server einen 
\textit{JWT (JSON Web Token)}, welcher für die weitere Verwendung dekodiert werden muss. Dieser kodierte 
Token wird im LocalStorage gespeichert, damit man sich nicht jedes mal neu einloggen muss, wenn man 
die Seite verlässt.
\begin{lstlisting}
    localStorage.setItem("user", JSON.stringify(response.data));
\end{lstlisting}
So wurde der Token in den LocalStorage der Seite gespeichert.


\textbf{TODO: Name Ref fauler neuner zu Token}

\subsection{Token dekodieren}

Das dekodieren von dem Token erfolgt in einer \textit{decodeToken} Methode. In dieser Methode wird der 
Token mittels \textit{post fetch} an den Server gesendet. War das Decoden des Tokens erfolgreich, 
erhalten wir als Antwort vom Server alle Daten zu dem eingeloggten User. Diese Daten werden dann in 
dem SessionStorage des Browsers gespeichert und werden somit beim schließen des Browsers wieder verworfen. 
Das dekodieren eines Tokens wird nach dem Login und bei dem Aufruf der Seite gemacht.
\begin{lstlisting}
    sessionStorage.setItem("data", JSON.stringify(data))
\end{lstlisting}
So erfolgt das Speichern der Tokendaten in den SessionStorage.

\subsection{Token Überprüfung}

Die Token selbst besitzen ein Ablaufdatum. Ist dieses Ablaufdatum erreicht, sind sie nicht mehr 
gültig und man wird ausgeloggt und muss sich wieder neu einloggen. Die Überprüfung dieses Ablaufdatum 
findet jedes mal statt, wenn die Seite neu aufgerufen wird. Wie auch schon beim Decoden eines Tokens,
ist dies auch ein \textit{post Fetch}, welches den Token an Server sendet. Ist das Ablaufdatum des
Token erreicht, gibt der Server als Antwort \textit{success:false} zurück. Ist dies der Fall ist der 
Token abgelaufen und man wird ausgeloggt. Ist der Token nicht abgelaufen gibt der Server die Antwort 
\textit{success:true} und bleibt eingeloggt. 

\subsection{Token entfernen}

Beim Ausloggen wird der Token aus dem LocalStorage entfernt. Die Information des dekodierten Token 
wird ebenfalls aus dem SessionStorage entfernt. Dies erfolgt ganz einfach mittels einer \textit{logout}
Methode, welche wie folgt aussieht:

\begin{lstlisting}
    logout() {
        localStorage.removeItem("user");
        sessionStorage.removeItem("data")
        sessionStorage.removeItem("validate")
    }    
\end{lstlisting}

Damit befindet sich nichts mehr im Local- und SessionStorage.
