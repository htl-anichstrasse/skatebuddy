\section{Benutzertoken}
\label{benutzer}

Der Benutzer kann sich mittels eines einfachen Formulars in die Website einloggen, als auch einen neuen 
Account erstellen. Bei erfolgreichem \nameref{login} oder \nameref{createAccount} erhält man vom Server einen 
\textit{JWT (\nameref{token})}, welcher für die weitere Verwendung dekodiert werden muss. Dieser kodierte 
Token wird im LocalStorage gespeichert, damit bei erneuten Aufruf der Seite das Login-Formular nicht neu 
ausgefüllt werden muss.

\begin{lstlisting}
    localStorage.setItem("user", JSON.stringify(response.data));
\end{lstlisting}
So wird der Token in den LocalStorage der Seite gespeichert.

\subsection{Token dekodieren}

Das Dekodieren von dem Token erfolgt in einer \textit{decodeToken} Methode. In dieser Methode wird der 
Token mittels \textit{post fetch} an den Server gesendet. War das Decoden des Tokens erfolgreich, 
beinhaltet die Antwort des Servers alle Daten zu dem eingeloggten User. Diese Daten werden dann im 
SessionStorage des Browsers gespeichert und werden somit beim schließen des Browsers wieder verworfen. 
Das Dekodieren eines Tokens wird nach dem Login und bei dem erneuten Aufruf der Seite gemacht.
\begin{lstlisting}
    sessionStorage.setItem("data", JSON.stringify(data))
\end{lstlisting}
So erfolgt das Speichern der Tokendaten in den SessionStorage.

\subsection{Token Überprüfung}

Die Token selbst besitzen ein Ablaufdatum. Ist dieses Ablaufdatum erreicht, sind sie nicht mehr 
gültig und werden aus dem LocalStorage geworfen, was zu einem automatischen ausloggen aus der Seite führt. Die Überprüfung dieses Ablaufdatums 
findet jedes mal statt wenn die Seite neu aufgerufen wird. Wie auch schon beim Decoden eines Tokens,
ist dies auch ein \textit{post Fetch}, welches den Token an Server sendet. Ist das Ablaufdatum des
Tokens erreicht, gibt der Server als Antwort \textit{success:false} zurück. 
Ist der Token nicht abgelaufen gibt der Server die Antwort \textit{success:true} und 
der Benutzer bleibt eingeloggt. 

\subsection{Token entfernen}

Beim Ausloggen wird der Token aus dem LocalStorage entfernt. Die Informationen des dekodierten Token 
werden ebenfalls aus dem SessionStorage entfernt. Dies erfolgt ganz einfach mittels einer \textit{logout}
Methode, welche wie folgt aussieht:

\begin{lstlisting}
    logout() {
        localStorage.removeItem("user");
        sessionStorage.removeItem("data")
        sessionStorage.removeItem("validate")
    }    
\end{lstlisting}

Damit befindet sich nichts mehr im Local- und SessionStorage.
