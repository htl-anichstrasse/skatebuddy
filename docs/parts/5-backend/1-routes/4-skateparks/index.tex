\label{/skateparks}

\subsubsection{GET /skateparks}

\begin{lstlisting}
    /skateparks
\end{lstlisting}

\textbf{Zweck der Route:} \\
Gibt  eine JSON liste mit allen vorhandenen Skateparks,  BilderID's und
Obstacles zurück.

\textbf{Authorization:} \\
Keine

\paragraph{Zusätzliche Parameter}
Keine

\paragraph{Body:}

Keinen

\subsubsection{Response}
\begin{code}
    \lstinline{Content-Type: JSON}
    \begin{lstlisting}
        [
            {
            "skateparkId": 1,
            "name": "Sane Skate Plaza Rum",
            "longitude": 11.448424,
            "latitude": 47.26836,
            "address": "Innsbruck Hallenbad O-Dorf, 6063 Innsbruck",
            "busstop": "Innsbruck Hallenbad O-Dorf",
            "rating": 3.5,
            "obstacles": [
                {
                    "ObstacleID": 8,
                    "Description": "Miniramp",
                    "difficulty": 4
                },
                ...
            ],
            "pictureIds": [
                {
                    "skateparkId": 1,
                    "picId": 1
                },
                ...
            ]
        },
    \end{lstlisting}
    \caption{Response der Get Skateparks Route}
\end{code}

\pagebreak

\subsubsection{GET /skateparks/:id}
\begin{lstlisting}
    /skateparks/2
\end{lstlisting}


\textbf{Zweck der Route} \\
Gibt einen Skatepark mit seinen zugehörigen Obstacles und Bilder ID's aus.




\paragraph{Zusätzliche Parameter}
\lstinline{id: int} (benötigt) \\
$\rightarrow$ Die eindeutige ID des gewünschten Skateparks


\textbf{Body:} \\
Keinen

\textbf{Authorization:} \\
Keinen

\paragraph{Response}
\begin{code}
    \lstinline{Content-Type: JSON}
    \begin{lstlisting}
    [
        {
        "skateparkId": 1,
        "name": "Sane Skate Plaza Rum",
        "longitude": 11.448424,
        "latitude": 47.26836,
        "address": "Innsbruck Hallenbad O-Dorf, 6063 Innsbruck",
        "busstop": "Innsbruck Hallenbad O-Dorf",
        "rating": 3.5,
        "obstacles": [
            {
                "ObstacleID": 8,
                "Description": "Miniramp",
                "difficulty": 4
            },
            {
                "ObstacleID": 7,
                "Description": "Manualpad",
                "difficulty": 6
            },
            {
                "ObstacleID": 2,
                "Description": "Flatrail",
                "difficulty": 7
            }
                    ]
        }
    ]
    \end{lstlisting}
    \caption{Response der Get Skatepark-Route}
\end{code}
\pagebreak


\subsubsection{Post /skateparks}

\begin{lstlisting}
    /skateparks
\end{lstlisting}

\textbf{Zweck der Route} \\
Mit dieser Route ist es möglich einen neuen Skatepark hinzuzufügen.





\paragraph{Zusätzliche Parameter}
\lstinline{id: int} (benötigt) \\
$\rightarrow$ Die eindeutige ID des gewünschten Skateparks


\textbf{Body:} \\
\begin{code}
    \lstinline{Content-Type: JSON}
    \begin{lstlisting}
        {
        "name": "Ibk1",
        "longitude": 4.23232,
        "latitude": 2.3232,
        "address": "dee",
        "busstop": "dwwd",
        "obstacles": [
            {
                "ObstacleID": 2,
                "difficulty": 0
            },
            {
                "ObstacleID": 5,
                "difficulty": 0
            }
        
    \end{lstlisting}
    \caption{Body der Post Skateparks-Route }
\end{code}

\textbf{Authorization:} \\
Keinen

\paragraph{Response }
\begin{code}
    \lstinline{Content-Type: JSON}
    \begin{lstlisting}
    [
        {
            "success": true,
            "message": "Successfully inserted"
        }
    ]
    \end{lstlisting}
    \caption{Response der Post Review Route}
\end{code}

\pagebreak


\subsubsection{DEL /skateparks/:id}

\begin{lstlisting}
    /skateparks/1
\end{lstlisting}

\textbf{Zweck der Route:} \\
Ein Skatepark mit einer bestimmten ID wird aus der Datenbank gelöscht.

\textbf{Authorization:} \\
Bearer

\paragraph{Zusätzliche Parameter}
\lstinline{id: int} (benötigt)
$\rightarrow$ Die eindeutige ID des gewünschten Skateparks.

\textbf{Body:}
Keinen


\subsubsection{Response)}
\begin{code}
    \lstinline{Content-Type: JSON}
    \begin{lstlisting}
        {
        "success": true,
        "message": "Successfully deleted"
    }
    \end{lstlisting}
    \caption{Response der delete Skateparks-Route }
\end{code}

\pagebreak


\subsubsection{PUT /skateparks/:id}

\begin{lstlisting}
    /skateparks/1
\end{lstlisting}

\textbf{Zweck der Route:} \\
Ein Skatepark mit einer bestimmten ID kann nach belieben geändert werden.
werden.

\textbf{Authorization:} \\
Keine

\paragraph{Zusätzliche Parameter}
\lstinline{id: int} (benötigt)
$\rightarrow$ Die eindeutige ID des gewünschten Parks..

\textbf{Body:}\\
\begin{code}
    \lstinline{Content-Type: JSON}
    \begin{lstlisting}
        {
            {
                "Title": "Sane Skate Plaza Rum",
                "newValue":"Nur Rum"
    
            }
    }
    \end{lstlisting}
    \caption{Body der Put Skateparks-Route}
\end{code}

\subsubsection{Response)}
\begin{code}
    \lstinline{Content-Type: JSON}
    \begin{lstlisting}
        {
            {
                "success": true,
                "message": "Successfully updated"
            }
    }
    \end{lstlisting}
    \caption{Response der Put Skateparks-Route}
\end{code}

\pagebreak

