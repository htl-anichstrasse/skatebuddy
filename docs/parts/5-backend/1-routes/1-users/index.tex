\label{/users}

\subsubsection{GET /users}

\begin{lstlisting}
    /users
\end{lstlisting}

\textbf{Zweck der Route:} \\
Gibt  eine JSON liste zurück mit allen vorhandenen Usern.

\textbf{Authorization:} \\
Bearer

\paragraph{Zusätzliche Parameter}

\textbf{Body:}
Keinen

\paragraph{Body:}

Keinen

\newpage

\subsubsection{Response}

\begin{code}[h]
    \lstinline{Content-Type: JSON}
    \begin{lstlisting}
[
    {
        "userId": 54,
        "name": "ww",
        "passwordhash": "$2b$10$6TnxbdU8GnROQK5oLAmEv.RcrlF9MCfC7Jlyv653L45FizJMGuz3u",
        "email": "w@w.w",
        "profilepictureId": null,
        "admin": 0
    },
    {
        "userId": 57,
        "name": "Alexander",
        "passwordhash": "$2b$10$rYPBy6ijGCb.Z/ZevSyJDuk4UXZnPlnnMKW4Y1X/zH1OUYNx93zUq",
        "email": "abertoni@tsn.at",
        "profilepictureId": null,
        "admin": 1
    }
]
\end{lstlisting}
    \caption{Response der get User-Route}
\end{code}



\pagebreak

\subsubsection{GET /users/:id}

\begin{lstlisting}
    /users/54
\end{lstlisting}

\textbf{Zweck der Route} \\
Gibt einen User mit einer bestimmten ID zurück.



\paragraph{Zusätzliche Parameter}
\lstinline{id: int} (benötigt) \\
$\rightarrow$ Die eindeutige ID des gewünschten Users

\textbf{Body:}
Keinen

\textbf{Authorization:} \\
Bearer

\paragraph{Response }

\begin{code}
    \lstinline{Content-Type: JSON}
    \begin{lstlisting}
    [
        {
            "userId": 54,
            "name": "ww",
            "passwordhash": "$2b$10$6TnxbdU8GnROQK5oL
                             AmEv.RcrlF9MCfC7Jlyv653L45FizJMGuz3u",
            "email": "w@w.w",
            "profilepictureId": null,
            "admin": 0
        },
    ]
    \end{lstlisting}
    \caption{Response für die get User-Route}
\end{code}

\pagebreak

\subsubsection{/users/decode}

\begin{code}
    \begin{lstlisting}
        /users/decode
    \end{lstlisting}
    \caption{Decode Route Beispiel}
\end{code}

\textbf{Zweck der Route:} \\
Im Body wird ein JSON Web token mitgesendet. Als Antwort
wird die Payload ausgeben.

\textbf{Authorization-Typ:} \\
Bearer


\paragraph{Zusätzliche Parameter}
Keine

\textbf{Body:}

\begin{code}
    \begin{lstlisting}

        "token": "eyJhbGciOiJIUzI1NJ.LCJleHAiOY5NTk3OTR9.k1p8njURt1GYHtio-Sg"
        
    \end{lstlisting}
    \caption{JWT Token}
\end{code}

\subsubsection{Response:}

\begin{code}
    \lstinline{Content-Type: Json}
    \begin{lstlisting}
        [
            "userId": 32,
            "name": "A",
            "email": "A@A",
            "profilepictureId": null,
            "iat": 1645295794,
            "exp": 1656959794
    ]
    \end{lstlisting}
    \caption{Response für die Decode Token Route}
\end{code}

\pagebreak

\subsubsection{GET /users/validate}

\begin{code}
    \begin{lstlisting}
        /users/validate
    \end{lstlisting}
    \caption{Beispiel Validate User Route}
\end{code}

\textbf{Zweck der Route:} \\
Im Body wird ein JWT-Token gesendet und der Server gibt zurück ob der Token
gültig ist oder nicht.

\textbf{Authorization:} \\
Bearer

\paragraph{Zusätzliche Parameter}
ID

\textbf{Body:}
\begin{code}
    \begin{lstlisting}

        "token": "eyJhbGciOiJIUzI1NJ.LCJleHAiOY5NTk3OTR9.k1p8njURt1GYHtio-Sg"
        
    \end{lstlisting}
    \caption{Body der Validate Route}
\end{code}


\subsubsection{Response}

\begin{code}
    \lstinline{Content-Type: JSON}
    \begin{lstlisting}
        {
            "success": true,
            "message": "Token Valid"
        }
    \end{lstlisting}
    \caption{Response für die Validate Token Route}
\end{code}

\pagebreak

\subsubsection{DEL /users/}

\begin{code}
    \begin{lstlisting}
        /users/1
    \end{lstlisting}
    \caption{Body der aktuellen Route}
\end{code}

\textbf{Zweck der Route:} \\
Ein User mit einer bestimmten ID wird aus der Datenbank gelöscht.

\textbf{Authorization:} \\
Bearer

\paragraph{Zusätzliche Parameter}
\lstinline{id: int} (benötigt)
$\rightarrow$ Die eindeutige ID des gewünschten Users

\textbf{Body:}
Keinen


\subsubsection{Response}

\begin{code}
    \lstinline{Content-Type: JSON}
    \begin{lstlisting}
        {
        "success": true,
        "message": "Successfully deleted"
    }
    \end{lstlisting}
    \caption{Response für die delete User Route}
\end{code}

\pagebreak


\subsubsection{PUT /users/:id}

\begin{code}
    \begin{lstlisting}
        /users/1
    \end{lstlisting}
    \caption{Beispiel für die Put User Route}
\end{code}

\textbf{Zweck der Route:} \\
Ein User mit einer bestimmten ID wird aus der Datenbank kann nach belieben geändert
werden.

\textbf{Authorization:} \\
Bearer

\paragraph{Zusätzliche Parameter}
\lstinline{id: int} (benötigt)
$\rightarrow$ Die eindeutige ID des gewünschten Users

\textbf{Body:}\\
\begin{code}
    \lstinline{Content-Type: JSON}
    \begin{lstlisting}
        {
            {
                "column": "name",
                "newValue":"hans"
    
            }
    }
    \end{lstlisting}
    \caption{Body der aktuellen Route}
\end{code}


\subsubsection{Response}

\begin{code}
    \lstinline{Content-Type: JSON}
    \begin{lstlisting}
        {
            {
                "success": true,
                "message": "Successfully updated"
            }
    }
    \end{lstlisting}
    \caption{Response der aktuellen Route}
\end{code}

\pagebreak

