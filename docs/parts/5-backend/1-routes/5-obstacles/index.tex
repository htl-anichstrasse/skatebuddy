\label{/obstacles}

\subsubsection{GET /obstacles}

\begin{lstlisting}
    /obstacles
\end{lstlisting}

\textbf{Zweck der Route:} \\
Gibt  eine JSON liste mit allen vorhandenen Obstacles zurück.

\textbf{Authorization:} \\
Keine

\paragraph{Zusätzliche Parameter}
Keine

\paragraph{Body:}

Keinen

\subsubsection{Response)}
\begin{code}
    \lstinline{Content-Type: JSON}
    \begin{lstlisting}
        [
            
        {
            "obstacleId": 1,
            "description": "Bowl"
        },
        {
            "obstacleId": 2,
            "description": "box"
        },
        {
            "obstacleId": 3,
            "description": "Mini Ramp"
        },
        ...
        ]
    \end{lstlisting}
    \caption{Response der get Obstacle-Route}
\end{code}

\pagebreak

\subsubsection{GET /obstacles/:id}

\begin{lstlisting}
    /obstacles/2
\end{lstlisting}

\textbf{Zweck der Route} \\
Gibt ein Obstacle mit einer bestimmten ID zurück.




\paragraph{Zusätzliche Parameter}
\lstinline{id: int} (benötigt) \\
$\rightarrow$ Die eindeutige ID des gewünschten Obstacles


\textbf{Body:} \\
Keinen

\textbf{Authorization:} \\
Keinen

\paragraph{Response }
\begin{code}
    \lstinline{Content-Type: JSON}
    \begin{lstlisting}
        {
            "obstacleId": 2,
            "description": "box"
        },
    \end{lstlisting}
    \caption{Response der Get Obstacle-Route}
\end{code}

\pagebreak


\subsubsection{Post /obstacles}

\begin{lstlisting}
    /obstacles/2
\end{lstlisting}

\textbf{Zweck der Route} \\
Mit dieser Route ist es möglich ein neues Obstacle zu erstellen.





\paragraph{Zusätzliche Parameter}
Keine


\textbf{Body:} \\
\begin{code}
    \lstinline{Content-Type: JSON}
    \begin{lstlisting}
        "description": "Mini Ramp"
    \end{lstlisting}
    \caption{Body der post Obstacle-Route}
\end{code}

\textbf{Authorization:} \\
Keinen

\paragraph{Response }
\begin{code}
    \lstinline{Content-Type: JSON}
    \begin{lstlisting}
    [
        {
            "success": true,
            "message": "Successfully inserted"
        }
    ]
    \end{lstlisting}
    \caption{Response der post Obstacle-Route}
\end{code}

\pagebreak


\subsubsection{DEL /obstacles/}

\begin{lstlisting}
    /obstacles/1
\end{lstlisting}

\textbf{Zweck der Route:} \\
Ein User mit einer bestimmten ID wird aus der Datenbank gelöscht.

\textbf{Authorization:} \\
Bearer

\paragraph{Zusätzliche Parameter}
\lstinline{id: int} (benötigt)
$\rightarrow$ Die eindeutige ID der gewünschten Review.

\textbf{Body:}
Keinen


\subsubsection{Response)}
\begin{code}
    \lstinline{Content-Type: JSON}
    \begin{lstlisting}
        {
        "success": true,
        "message": "Successfully deleted"
    }
    \end{lstlisting}
    \caption{Response der delete Obstacle-Route}
\end{code}

\pagebreak


\subsubsection{PUT /obstacles/:id}

\begin{lstlisting}
    /obstacles/1
\end{lstlisting}

\textbf{Zweck der Route:} \\
Ein Obstacle mit einer bestimmten ID kann nach belieben geändert werden.

\textbf{Authorization:} \\
Keine

\paragraph{Zusätzliche Parameter}
\lstinline{id: int} (benötigt)
$\rightarrow$ Die eindeutige ID des gewünschten Obstacles.

\textbf{Body:}\\
\begin{code}
    \lstinline{Content-Type: JSON}
    \begin{lstlisting}
        {
            {
                "Title": "Mini Ramp",
                "newValue":"Kleine Rampe"
    
            }
    }
    \end{lstlisting}
    \caption{Body der pur Obstacle-Route}
\end{code}

\subsubsection{Response}

\lstinline{Content-Type: JSON}
\begin{lstlisting}
    {
        {
            "success": true,
            "message": "Successfully updated"
        }
}
\end{lstlisting}

\pagebreak

