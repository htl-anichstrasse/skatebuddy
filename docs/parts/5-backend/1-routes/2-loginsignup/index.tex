
\subsubsection{POST /register}

\begin{lstlisting}
        /register
    \end{lstlisting}


\textbf{Zweck der Route:} \\
Mit dieser Route kann man sich registrieren. Der Server
sendet dann eine Nachricht mit einem Erfolg oder nicht Erfolg.
Bei dieser Route erhält der User einen gültigen Bearer Token.

\textbf{Authorization:} \\
Keine

\paragraph{Zusätzliche Parameter}
Keine

\textbf{Body:}
\begin{code}
    \lstinline{Content-Type: JSON}
    \begin{lstlisting}
        {
            {
        "name": "Test",
        "password": "Test",
        "email": "test@test.com"
    }  
    }
    \end{lstlisting}
    \caption{Body der Register-Route}
\end{code}

\subsubsection{Response)}

\begin{code}
    \lstinline{Content-Type: JSON}
    \begin{lstlisting}
        {
            {
                "success": true,
                "token": "eyJhbGciOiJIUzI1NJ.LCJleHAiOY5NTk3OTR9.k1p8njURt1GYHtio-Sg"
            }
    }
    \end{lstlisting}
    \caption{Response der Register-Route}
\end{code}

\pagebreak

\subsubsection{POST /login}

\begin{lstlisting}
    /login
\end{lstlisting}


\textbf{Zweck der Route:} \\
Mit dieser Route kann man sich einloggen. Der Server
sendet dann eine Nachricht mit einem Erfolg oder nicht Erfolg.
Bei dieser Route erhält der User einen gültigen Bearer Token.

\textbf{Authorization:} \\
Keine

\paragraph{Zusätzliche Parameter}
Keine

\textbf{Body:}
\begin{code}
    \lstinline{Content-Type: JSON}
    \begin{lstlisting}
    {
        {
            "name": "Test",
            "password": "Test",
            "email": "test@test.com"
        }  
}
\end{lstlisting}
    \caption{Beispiel Body Login-Route}
\end{code}


\subsubsection{Response)}
\begin{code}
    \lstinline{Content-Type: JSON}
    \begin{lstlisting}
        {
            {
                "success": true,
                "token": "eyJhbGciOiJIUzI1NJ.LCJleHAiOY5NTk3OTR9.k1p8njURt1GYHtio-Sg"
            }
    }
    \end{lstlisting}
    \caption{Response der Login-Route}
\end{code}

\pagebreak