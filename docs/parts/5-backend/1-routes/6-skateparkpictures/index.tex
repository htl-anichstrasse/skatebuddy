\label{/skateparkpictures}

\subsubsection{GET /skateparkpictures}

\begin{lstlisting}
    /skateparkpictures
\end{lstlisting}

\textbf{Zweck der Route:} \\
Gibt  eine JSON liste mit allen vorhandenen Skateparkpicture-ID's zurück.

\textbf{Authorization:} \\
Keine

\paragraph{Zusätzliche Parameter}
Keine

\paragraph{Body:}

Keinen

\subsubsection{Response)}

\lstinline{Content-Type: JSON}
\begin{lstlisting}
    [
    {
        "skateparkId": 1,
        "picId": 1
    },
    {
        "skateparkId": 1,
        "picId": 2
    },
    {
        "skateparkId": 1,
        "picId": 3
    },
    ... 
    ]
\end{lstlisting}

\pagebreak

\subsubsection{GET /skateparkpictures/:id/:pid}

\begin{lstlisting}
    /obsskateparkpicturestacles/2/1
\end{lstlisting}

\textbf{Zweck der Route} \\
Ruft ein Bild vom Server mit einer Park und Bild Nummer ab. Der Server Antwortet
dannach mit den Binärdaten des Bildes.




\paragraph{Zusätzliche Parameter}
\lstinline{id: int} (benötigt) \\
$\rightarrow$ Die eindeutige ID des gewünschten Skateparks und des Bildes


\textbf{Body:} \\
Keinen

\textbf{Authorization:} \\
Keinen

\paragraph{Response }

\lstinline{Content-Type: JSON}
\begin{lstlisting}
    Binaerdaten des Bildes
\end{lstlisting}
\pagebreak

\subsubsection{GET /skateparkpictures/:id}

\begin{lstlisting}
    /s#skateparkpictures/3
\end{lstlisting}

\textbf{Zweck der Route} \\
Gibt eine Liste aller Obstacles in einem Park zurück.




\paragraph{Zusätzliche Parameter}
\lstinline{id: int} (benötigt) \\
$\rightarrow$ Die eindeutige ID des gewünschten Skateparks.


\textbf{Body:} \\
Keinen

\textbf{Authorization:} \\
Keinen

\paragraph{Response }

\lstinline{Content-Type: JSON}
\begin{lstlisting}
    [
        4,
        6    
    ]
\end{lstlisting}