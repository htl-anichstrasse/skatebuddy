\label{/reviews}

\subsubsection{GET /reviews}

\begin{lstlisting}
    /reviews
\end{lstlisting}

\textbf{Zweck der Route:} \\
Gibt  eine JSON liste zurück mit allen vorhandenen Reviews.

\textbf{Authorization:} \\
Keine

\paragraph{Zusätzliche Parameter}
Keine

\paragraph{Body:}

Keinen

\subsubsection{Response)}

\lstinline{Content-Type: JSON}
\begin{lstlisting}
    [
        {
            "reviewId": 122,
            "skateparkId": 2,
            "userId": 58,
            "rating": 5,
            "title": "Grosse Bowl",
            "content": "Einfach nur perfekt.",
            "username": "Philipp"
        },
        {
            "reviewId": 121,
            "skateparkId": 1,
            "userId": 58,
            "rating": 3,
            "title": "Keine Bowl",
            "content": "Es gibt zwar eine Miniramp, aber keine Bowl. Trotzdem guter Park.",
            "username": "Philipp"
        },
        {
            "reviewId": 120,
            "skateparkId": 1,
            "userId": 57,
            "rating": 4,
            "title": "Guter Park",
            "content": "Ein sehr guter Park mit einer grossen und abwechslungsreichen Anzahl an Hindernissen.",
            "username": "Alexander"
        }
]
\end{lstlisting}

\pagebreak

\subsubsection{GET /reviews/:id}

\begin{lstlisting}
    /reviews/2
\end{lstlisting}

\textbf{Zweck der Route} \\
Gibt einen Review mit zu einem bestimmten Skatepark zurück.



\paragraph{Zusätzliche Parameter}
\lstinline{id: int} (benötigt) \\
$\rightarrow$ Die eindeutige ID des gewünschten Skateparks


\textbf{Body:} \\
Keinen

\textbf{Authorization:} \\
Keinen

\paragraph{Response }

\lstinline{Content-Type: JSON}
\begin{lstlisting}
    [
        {
            "reviewId": 122,
            "skateparkId": 2,
            "userId": 58,
            "rating": 5,
            "title": "Grosse Bowl",
            "content": "Einfach nur perfekt.",
            "username": "Philipp"
        }
]
\end{lstlisting}

\pagebreak


\subsubsection{Post /reviews}

\begin{lstlisting}
    /reviews/2
\end{lstlisting}

\textbf{Zweck der Route} \\
Mit dieser Route ist es möglich eine neue Review zu einem bestimmten Skatepark hinzuzufügen.




\paragraph{Zusätzliche Parameter}
\lstinline{id: int} (benötigt) \\
$\rightarrow$ Die eindeutige ID des gewünschten Skateparks


\textbf{Body:} \\
\lstinline{Content-Type: JSON}
\begin{lstlisting}
    {
        "parkid": "2",
        "userid": "10",
        "rating": "5",
        "title": "Cool",
        "content": "Nice"
    }
\end{lstlisting}

\textbf{Authorization:} \\
Keinen

\paragraph{Response }

\lstinline{Content-Type: JSON}
\begin{lstlisting}
[
    {
        "success": true,
        "message": "Successfully inserted"
    }
]
\end{lstlisting}

\pagebreak


\subsubsection{DEL /reviews/}

\begin{lstlisting}
    /users/1
\end{lstlisting}

\textbf{Zweck der Route:} \\
Eine Review mit einer bestimmten ID wird aus der Datenbank gelöscht.

\textbf{Authorization:} \\
Bearer

\paragraph{Zusätzliche Parameter}
\lstinline{id: int} (benötigt)
$\rightarrow$ Die eindeutige ID der gewünschten Review.

\textbf{Body:}
Keinen


\subsubsection{Response)}

\lstinline{Content-Type: JSON}
\begin{lstlisting}
    {
    "success": true,
    "message": "Successfully deleted"
}
\end{lstlisting}

\pagebreak


\subsubsection{PUT /reviews/:id}

\begin{lstlisting}
    /reviews/1
\end{lstlisting}

\textbf{Zweck der Route:} \\
Eine Review mit einer bestimmten ID kann nach belieben geändert werden.
werden.

\textbf{Authorization:} \\
Keine

\paragraph{Zusätzliche Parameter}
\lstinline{id: int} (benötigt)
$\rightarrow$ Die eindeutige ID der gewünschten Review.

\textbf{Body:}\\
\lstinline{Content-Type: JSON}
\begin{lstlisting}
    {
        {
            "Title": "Alter Titel",
            "newValue":"Neuer Titel"

        }
}
\end{lstlisting}

\subsubsection{Response)}

\lstinline{Content-Type: JSON}
\begin{lstlisting}
    {
        {
            "success": true,
            "message": "Successfully updated"
        }
}
\end{lstlisting}


\pagebreak

