\label{/reviews}

\subsubsection{GET /reviews}

\begin{lstlisting}
        /reviews
\end{lstlisting}

\textbf{Zweck der Route:} \\
Gibt  eine JSON liste zurück mit allen vorhandenen Reviews.

\textbf{Authorization:} \\
Keine

\paragraph{Zusätzliche Parameter}
Keine

\paragraph{Body:}

Keinen

\subsubsection{Response}

\begin{code}
    \lstinline{Content-Type: JSON}
    \begin{lstlisting}
        [
            {
                "reviewId": 122,
                "skateparkId": 2,
                "userId": 58,
                "rating": 5,
                "title": "Grosse Bowl",
                "content": "Einfach nur perfekt.",
                "username": "Philipp"
            },
            {
                "reviewId": 121,
                "skateparkId": 1,
                "userId": 58,
                "rating": 3,
                "title": "Keine Bowl",
                "content": "Es gibt zwar eine Miniramp, aber keine Bowl. Trotzdem guter Park.",
                "username": "Philipp"
            },
    ]
    \end{lstlisting}
    \caption{Response der Get Review Route }
\end{code}

\pagebreak

\subsubsection{GET /reviews/:id}

\begin{lstlisting}
    /reviews/2
\end{lstlisting}

\textbf{Zweck der Route} \\
Gibt einen Review mit zu einem bestimmten Skatepark zurück.



\paragraph{Zusätzliche Parameter}
\lstinline{id: int} (benötigt) \\
$\rightarrow$ Die eindeutige ID des gewünschten Skateparks


\textbf{Body:} \\
Keinen

\textbf{Authorization:} \\
Keinen

\paragraph{Response }
\begin{code}
    \lstinline{Content-Type: JSON}
    \begin{lstlisting}
        [
            {
                "reviewId": 122,
                "skateparkId": 2,
                "userId": 58,
                "rating": 5,
                "title": "Grosse Bowl",
                "content": "Einfach nur perfekt.",
                "username": "Philipp"
            }
    ]
    \end{lstlisting}
    \caption{Response der Get Review-Route}
\end{code}

\pagebreak


\subsubsection{Post /reviews}


\begin{lstlisting}
        /reviews/2
    \end{lstlisting}

\textbf{Zweck der Route} \\
Mit dieser Route ist es möglich eine neue Review zu einem bestimmten Skatepark hinzuzufügen.




\paragraph{Zusätzliche Parameter}
\lstinline{id: int} (benötigt) \\
$\rightarrow$ Die eindeutige ID des gewünschten Skateparks


\textbf{Body:} \\
\begin{code}
    \lstinline{Content-Type: JSON}
    \begin{lstlisting}
        {
            "parkid": "2",
            "userid": "10",
            "rating": "5",
            "title": "Cool",
            "content": "Nice"
        }
    \end{lstlisting}
    \caption{Body der Post Review-Route}
\end{code}

\textbf{Authorization:} \\
Keinen

\paragraph{Response }
\begin{code}
    \lstinline{Content-Type: JSON}
    \begin{lstlisting}
    [
        {
            "success": true,
            "message": "Successfully inserted"
        }
    ]
    \end{lstlisting}
    \caption{Response der Post Review-Route}
\end{code}

\pagebreak


\subsubsection{DEL /reviews/}


\begin{lstlisting}
        /reviews/1
    \end{lstlisting}

\textbf{Zweck der Route:} \\
Eine Review mit einer bestimmten ID wird aus der Datenbank gelöscht.

\textbf{Authorization:} \\
Bearer

\paragraph{Zusätzliche Parameter}
\lstinline{id: int} (benötigt)
$\rightarrow$ Die eindeutige ID der gewünschten Review.

\textbf{Body:}
Keinen


\subsubsection{Response}

\begin{code}
    \lstinline{Content-Type: JSON}
    \begin{lstlisting}
        {
        "success": true,
        "message": "Successfully deleted"
    }
    \end{lstlisting}
    \caption{Response der delete-Route}
\end{code}

\pagebreak


\subsubsection{PUT /reviews/:id}

\begin{lstlisting}
        /reviews/1
    \end{lstlisting}


\textbf{Zweck der Route:} \\
Eine Review mit einer bestimmten ID kann nach belieben geändert werden.
werden.

\textbf{Authorization:} \\
Keine

\paragraph{Zusätzliche Parameter}
\lstinline{id: int} (benötigt)
$\rightarrow$ Die eindeutige ID der gewünschten Review.

\textbf{Body:}\\
\begin{code}
    \lstinline{Content-Type: JSON}
    \begin{lstlisting}
    {
            {
                "Title": "Alter Titel",
                "newValue":"Neuer Titel"
    
            }
    }
    \end{lstlisting}
    \caption{Body der Put Review-Route}
\end{code}

\subsubsection{Response}
\begin{code}
    \lstinline{Content-Type: JSON}
    \begin{lstlisting}
    {
            {
                "success": true,
                "message": "Successfully updated"
            }
    }
    \end{lstlisting}
    \caption{Response der Put Review Route}
\end{code}


\pagebreak

