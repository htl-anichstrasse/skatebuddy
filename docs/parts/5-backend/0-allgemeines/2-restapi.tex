\section{Rest-Api}
\label{restapi}

\textbf{Rest} steht für REpresentational State Transfer, \textbf{API} für Application
Programming Interface. Die Rest Api hat die Aufgabe den CLient bzw. die App
mithilfe von sogenannten \textbf{Rest-Routes} zugriff auf die Datenbank zu geben.
Im Prinzip kann jeder Computer mit Internet Zugang auf die API zugreifen, insofern
sie die vorgefertigten \textbf{HTTP-Anfragen} an den Server senden. Wie genau diese
Anfragen aussehen ist im Kapitel \textbf{Rest-Routes} beschrieben \ref*{}.

\begin{figure}[H]
    \begin{center}
        \includegraphics*[]{Backend/Rest APi..png}

    \end{center}
\end{figure}


\subsection{Authorization}

Damit sensible Daten wie zum Beispiel die User Daten nicht für
alle Pc's frei zugänglich sind werden bestimmte \textit{Routes} mit einer
\textit{Bearer Authorization} gesichert. Um die Routes benutzen zu können muss
im Header der HTTP Anfrage ein Ein
