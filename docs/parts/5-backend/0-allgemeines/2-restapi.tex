\section{Rest-Api}
\label{restapi}

\textbf{Rest} steht für REpresentational State Transfer, \textbf{API} für Application
Programming Interface. Die Rest Api hat die Aufgabe den CLient bzw. die App
mithilfe von sogenannten \textbf{Rest-Routes} zugriff auf die Datenbank zu geben.
Im Prinzip kann jeder Computer mit Internet Zugang auf die API zugreifen, insofern
sie die vorgefertigten \textbf{HTTP-Anfragen} an den Server senden. Wie genau diese
Anfragen aussehen ist im Kapitel \textbf{Rest-Routes} beschrieben \ref*{}.

\begin{figure}[H]
    \begin{center}
        \includegraphics*[]{Backend/Rest APi..png}
    \end{center}
\end{figure}


\subsection{Authorization}

Damit sensible Daten wie zum Beispiel die User Daten nicht für
alle Pc's frei zugänglich sind werden bestimmte \textit{Routes} mit einer
\textit{Bearer Authorization} gesichert. Um die Routes benutzen zu können muss
im Header der HTTP Anfrage ein \texttt{Authorization} Eintrag mitgesendet werden.
Falls keiner mitgesendet wird sendet die API eine Message mit den Inhalt
\textbf{Unauthorized}.

Nur Benutzer die einen sogenannten \textit{Bearer-Token} besitzen haben also zugriff
auf die Ressourcen.

Einen solchen Token erhält man entweder wenn man sich Registriert siehe (api/register) oder wenn sich
Anmeldet siehe (api/login).

Wie so ein Bearer Token aussieht sehen sie hier:

\texttt{eyJhbGciOiJIUzI1NiIsInR5cCI6IkpXVCJ9.eyJ1c2VySWQiOjMyLCJuYW1lIjoiQSIsImVt
    YWlsIjoiQUBBIiwicHJvZmlsZXBpY3R1cmVJZCI6bnVsbCwiaWF0IjoxNjQ1Mjk1Nzk0LCJleHAiOjE2N
    TY5NTk3OTR9.k1p8njU8ra6Rt16nZTTCmnkrhK3H8MygbwGYHtio-Sg}

Als Bearer Token haben wir uns für den \textit{Json Web Token (JWT)} entschieden.

\subsubsection{Json Web Token}
Der \textit{JWT} wird in der Regel für Authentifizierungsvorgängen verwendet. Es ist ebenfalls
möglich, in ihm Informationen zu speichern. Die Informationen können verschlüsselt werden
und somit sichergestellt werden, wer der Absender ist und ob dieser die benötigten Zugriffsrechte
hat. Ein solcher Token besteht aus drei Teilen.

\paragraph{Header}
Der Header besteht aus zwei Teilen und liefert Informationen über den Token. Welcher Typ und
welcher Verschlüsselungsalgorithmus angewendet wurde.

\paragraph{Payload}
In der Payload wird die tatsächliche Information des Tokens gespeichert. Die Informationen
werden als \textbf{Key/Value Paare} bereitgestellt.

\paragraph{Signature}
Die Signatur des Schlüssels wird mithilfe der \textit{Base-64 Kodierung} der Headers
der Payload und der angegebenen Signatur-/Verschlüsselungsmethode erstellt. Damit die Signatur
gültig ist wird ebenfalls noch ein geheimer Schlüssel benötigt dem nur der Ursprungsquelle bekannt
ist. Somit wird sichergestellt das der Token nicht verändert wurde.

\begin{figure}[H]
    \begin{center}
        \includegraphics*[width=0.7\textwidth]{Backend/JWTToken.png}
    \end{center}
    \url{https://medium.com/swlh/understand-the-concept-of-jwt-json-web-tokens-4ee18682a583}
\end{figure}



\cite{JWT}


