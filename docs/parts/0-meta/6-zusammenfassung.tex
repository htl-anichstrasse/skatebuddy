\section*{Zusammenfassung}

Neue Skater haben oft keine Auskunft über die lokalen Parks bzw. wissen gar nicht wie viele
Skateparks es mittlerweise in Tirol gibt.


In der App wollen wir alle umliegenden Parks auflisten und eine kleine
Übersicht bieten (Hindernisse mit Schwierigkeitsstufen, Skateshops in der Nähe, Bewertungen der
Benutzer, usw.). Außerdem ist es den Usern möglich Videos ihrer Tricks hochzuladen und zu teilen.

Bis zu 1,3 Millionen Tonnen Essen werden jedes Jahr einfach weggeworfen. Ein großer Teil dieser Verschwendung stammt aus nicht verkauften Lebensmitteln in Restaurants und im Einzelhandel.~\cite{depta2018} Diese Diplomarbeit hat das Ziel eine Linderung für dieses schwerwiegende Problem der heutigen Zeit durch die Nutzung moderner Technologie zu finden.

Hinter dem Namen \textbf{Sokka} verbirgt sich ein ökologisches Bestellsystem für Kantinen, das den technischen Ansprüchen des 21. Jahrhunderts gerecht wird. \textit{Sokka} erlaubt es Kunden einer Kantine, ihr gewünschtes Essen bereits am Vortag über eine einfache Smartphone-App zu bestellen, um dann am darauffolgenden Tag ihre Bestellung durch das Vorzeigen eines von der App generierten Codes abholen zu können.

Wenn alle Kunden einer Kantine \textit{Sokka} nutzen, ist es den Köchen der Kantine möglich genau einzuschätzen, welche Menge an Essen sie für den kommenden Tag zubereiten müssen. So bleibt im Optimalfall kein Essen übrig und es muss nichts weggeworfen werden.

\textit{Sokka ist ein Open-Source-Projekt (FOSS). Der Projektquellcode ist online auf GitHub unter den Lizenzbestimmungen der GNU-GPLv3-Lizenz verfügbar.\\github.com/htl-anichstrasse/sokka}

\newpage